\documentclass{scrartcl}

\usepackage{hyperref}
\usepackage{amsfonts}
\usepackage{amssymb}
\usepackage{amsmath}
\usepackage[utf8]{inputenc}
\usepackage[german]{babel}
\usepackage{xstring}

\title{Lineare Algebra I\\Mitschrieb}
\author{Florian Kramer}

\newcommand{\lb}{\lambda}
\newcommand{\mlb}{\(\lb\)}
\newcommand{\R}{\mathbb{R}}
\newcommand{\N}{\mathbb{N}}
\newcommand{\mR}{\(\mathbb{R}\)}
\newcommand{\mN}{\(\mathbb{N}\)}
\newcommand{\Rn}{\mathbb{R}^n}
\newcommand{\mRn}{\(\mathbb{R}^n\)}
\newcommand{\al}{\alpha}
\newcommand{\be}{\beta}
\newcommand{\vtwo}[2]{\begin{pmatrix}#1 \\ #2 \end{pmatrix}}
\newcommand{\vthree}[3]{\begin{pmatrix}#1 \\ #2 \\ #3 \end{pmatrix}}
\newcommand{\ve}[1]{{\begin{pmatrix}#1 \end{pmatrix}}}
\renewcommand{\v}{\ve}


\renewcommand{\baselinestretch}{1.5}

\begin{document}
\maketitle
\tableofcontents
\pagebreak


\section{Einf\"uhrung}
\begin{itemize}
\item{Das Wort Algebra stammt aus dem arabischen "al-jabr".}
\item{Allgemein ist Algebra die Lehre der mathematischen Symbole und deren Manipulation.}
\item{Lineare Algebra: Insbesondere lineare Gleichungen}
\end{itemize}

\subsection{Aufbau der Vorlesung}
\begin{enumerate}
\item{Lineare Gleichungssysteme und der n-dimensionale reellen Raum}
\item{Grundlegende Objekte}
\item{Gruppen, Ringe, K\"orper}
\item{Vektorr\"aume und lineare Abbildungen}
\item{Determinaten}
\item{Eigenwerte und Normalformen}
\end{enumerate}

\subsection{Beispiel: Der Google Pagerank}
Gegeben 4 Seiten, mit Verlinkungen zwischen den Seiten. Von einer nicht  verlinkten Seite wechselt man zuf\"llig auf eine andere Seite. Der user startet an einer zuf\"lligen Stelle und folgt von dort einem zuf\"alligen link auf eine andere Seite.Zus\"atzlich wird immer mit Wahrscheinlichkeit \((1-d), d \in [0, 1]\) auf eine beliebige Website gewechselt.\\
Die wichtigste Site ist nun die, auf welcher ein Benutzer sich mit der h\"ochsten Wahscheinlichkeit aufh\"alt.\\
\(
p(\delta_1) = \frac{1-d}{N} + d(\frac{p(\delta_2)}{1}, \frac{p(\delta_5)}{4})\\
p(\delta_2) = \frac{1-d}{N} + d(\frac{p(\delta_1)}{3}, \frac{p(\delta_5)}{4})\\
\vdots
\)\\
Zur Berechnung von \(p(\delta_j), j \in \{1..5\}\) gibt es Methoden aus der linearen Algebra.


\section{Lineare Gleichungssyteme und der n-dimensionale reelle Raum}
\begin{itemize}
\item{Descartes f\"uhrte "Koordinaten" ein in der Geometrie ein, also Zahlensysteme. Das f\"uhrte dazu, das man nun leichter rechnen kann.}
\item{Wir benutzen hier die reellen Zahlen (mit den \"ublichen Rechenregeln, also f\"ur die Addition : 
\begin{itemize}
\item{\((x + y) + z = x + (y + z)\)}\\
\item{\(0 + x = x + 0 = x\)}
\item{Es gibt f\'ur jedes x ein y mit \(x + y = 0\), wir nennen dieses y das additiv inverse zu x ("-x").}
\item{\(x + y = y + x\)}
\end{itemize}
Und f\"ur multiplikation:
\begin{itemize}
\item{\(\lambda (x + y) = \lambda x + \lambda y\)}
\item{\((\lambda + \mu) x = \lambda x + \mu x)\)}
\item{\(\lb(\rho\mu)=(\lb\rho)\mu\)}
\item{\(1x = x\)}
\end{itemize}
}
\item{Weiteres brauchen wir die nat\"urlichen Zahlen, die 1,2,3\dots}
\end{itemize}

\subsection{Der \(\R^n\)}
F\"ur gegebenes \(n \in \N\) definieren wir:\\
\(\R^n = \{x = (x_1, x_2, \dots, x_n): x_1, \dots, x_n \in \R\}\)\\
\((x_1, \dots, x_n)\) ist dabei ein geordnetes n-Tupel, die Reihenfolge beim Vergleich Elemente dieser Art ist wichtig.\\
F\"ur \(x, y, \in \R : x = y \Leftrightarrow x_1 = y_1, x_2 = y_2, \dots x_n, = y_n\)\\
Wir nennen diese n-Tupel auch Vektoren im \mRn.\\
Mit \(\R^0\) bezeichnen wir die Menge \(\{0\}\), welche nur das Nullelement enth\"alt.\\
Die Rechenregeln \"ubertragen sich nun von \mR. Wir schreiben\\
\(x + y = (x_1 + y_1, \dots, x_n + y_n)\) f\"ur \(x, y \in \Rn\) (Vektoradition)\\
\(\lb x = (\lb x_1, \dots, \lb x_n)\) (Skalarmultiplikation)

\subsection{Lineare Gleichungssyteme}
Eine lineare Gleichung ueber \mR ist ein Ausdruck der Form:\\
\(
\alpha_1x_1 + \alpha_2 x_2 + \dots \alpha_n x_n = \beta\)\\
Fuer reele Zahlen \(\beta, \alpha_1, \dots, \alpha_n \in \R\). Einen Vektor, \(\xi = \{\xi_1, \dots, \xi_n) \in \Rn\) nennen wir Loesung, wenn die reelen Zahleen  , \(\xi_1, \dots, \xi_n\) eingesetzt in \(x_1, \dots, x_n\) die Gleichung erf\"ullen.\\
Ein lineares Gleichungssystem G ist ein System \\
\(
a_11 x_1 + a_12 + x_2 \dots a_1n x_n = b_1\\
a_21 x_1 + a_22 + x_2 \dots a_2n x_n = b_1\\
\vdots\\
a_m1 x_1 + a_m2 + x_2 \dots a_mn x_n = b_1\\
\)\\
In Kurzform = \(\sum_{j=1}^n a_{i,j} x_j = b_i \forall i\in\{1,\dots,m\}\)\\
oder, noch k\"urzer, in Matrixschreibweise\\
\(Ax=b\)\\
Wobei \(A\) eine Matrix ist mit Eintr\"agen \(a_{i,j}\), wir schreiben\\
\(
A = 
\begin{pmatrix}
a_{1, 1} & \dots & a_{1, n}\\
 & \vdots\\
a_{m, 1} & \dots & a_{m, n}
\end{pmatrix}
\)\\
\(Ax\) f\"ur \(x \in \Rn\) ist dann eine Kurzform f\"ur \(\sum_{i=1}^na_{ij}x_j\) mit einem Vektor \(x = (x_1, \dots, x_n) \in \Rn\). Das Ergebnis ist ein Vektor \(b = (b_1, \dots, b_m) \in \R^m\) f\"ur eine Matrix \(A\) mit m Zeilen und n Spalten.\\
Der Vektor b hei\ss{}t rechte Seite des linearen Gleichungssystems, A hei\ss{}t Koeffizienten Matrix des linearen Gleichungssystems. Eine Spalte / Zeile von A kann mit einem Vektor im \(\R^m\) bzw. im \(\R^n\) identifiziert werden. Wir sprechen von Spalten- / Zeilen Vektoren der Matrix a.\\
Eine Matrix mit m Zeilen und n Spalten nennen wir mxn - Matrix. Fuer \(x  \in \Rn \), A eine mxn - Matix und B eine lxm - Matrix gilt die Rechenregel \(BAx = B(Ax)\). Ein Gleichugnssystem \(Ax=b\) heisst homogen, falls b der Nullvektor \((0, \dots, 0)\) ist und quadratisch fuer \(m = n\) (eine quadratische Matrix A).\\
\subsubsection{Definition: Normalform}
Ein Gleichungssytstem \(Ax=b\) ist min Normalform, falss A die Gestalt \\
\(
\begin{pmatrix}
1 & 0 & a_{1,k+1}& \dots & a_{1, n}\\
0 & 1 & a_{m, k+1} & \dots & a_{m, n}\\
0 & 0 & 0 & 0 & 0\\
0 & 0 & 0 & 0 & 0\\
0 & 0 & 0 & 0 & 0\\
\end{pmatrix}
\) (fuer k=2)\\
fuer ein \(k \in \N_0\)\\
Beispiele:\\
\(
\begin{pmatrix}
1 & 0 & 3\\
0 & 1 & 4\\
0 & 0 & 0\\
0 & 0 & 0\\
\end{pmatrix}
\) Ist in Normalform f\"ur \(k = 2\).\\
\(
\begin{pmatrix}
1 & 0 & 0\\
0 & 1 & 0\\
0 & 0 & 1\\
\end{pmatrix}
\) Ist in Normalform f\"ur \(k = 3\).\\
\(
\begin{pmatrix}
0 & 0\\
0 & 0\\
\end{pmatrix}
\) Ist in Normalform f\"ur \(k = 0\).\\
k hei\ss{}t Rang der Matrix A (bzw. des Gleichungssytems). Es gilt \\
\(0 \le k \le min(m, n)\\ \)
Ein Gleichungssystem ist genau dann L\"osbar,  wenn gilt:\\
\(
b_{k+1} = b_{k+2} = \dots = b_m = 0
\)\\
In diesem Fall l\"asst sich eine L\"osung \(\xi \in \Rn\) bestimmen,  indem man \(\xi_{k+1}, \dots, \xi_n\) beliebig w\"ahlt, und danach \(\xi_i = b_i -  \sum_{j=k+1}^n a_{i,j} \xi_j \forall i\in\{1,\dots,n\}\) w\"ahlt.\\
Denn f\"ur Zeile \(i, i=k+1, \dots, n\) lautet das Gleichungssystem \(0x_1 + \dots + 0x_n = b_i = 0\) und \\
F\"ur Zeile \(i, i =1, \dots, k\) \\
\(a_{i, i} x_i + \sum_{j=k+1}^n a_{i,j} x_j = b_i\)\\
Beispiele:
\(
\begin{pmatrix}
1 & 0 & 3\\
0 & 1 & 4\\
0 & 0 & 0\\
0 & 0 & 0\\
\end{pmatrix} x = b = 
\begin{pmatrix}
1\\1\\0\\0
\end{pmatrix}
\)\\
W\"ahle \(x_3 = 1\). Dann folgt daraus \(x_2 = -3\) und \(x_1 = -2\)\\
Wir sagen die L\"osungsmenge ist \\
\(\{(b_1 - \sum_{j=k+1}^na_{1j}\xi_j), \dots,  (b_k - \sum_j={k+1}^na_{kj}\xi_j), \xi_{k+1}, \dots, \xi_n : \xi_{k+1}, \dots, \xi_n \in \R\}\).\\
Wir nennen eine solche Mengen (n-k) parametrig.

\subsubsection{Lemma 0.1}
Sei A eine \(mxn\) -Matrix mit Rang k. Dann gilt \(k=n\)  genau dann, wenn alle Gleichungssysteme mit A h\"ochstens eine L\"osung haben, und k = m, genau dann, wenn alle Gleichungssyteme mit A l\"osbar sind.\\
Beweis: klar aus der Darstellung.\\

\subsubsection{Zeilenoperationen}
Eine Zeilenoperation macht aus dem Gleichungssystem ein neues Gleichungssytem durch Multiplikation  der i-ten Zeile mit einer Zahl \(\lb \in \R \setminus 0\) oder durch addieren des \mlb -fachen der i-ten Zeile zur j-ten Zeile \((i \neq j)\). Wir bezeichnen diese Operationen mit \(Z_i^\lb\) bzw. \(Z_{i,j}^\lb\).\\
Bemerkung: Die Zeilenoperationen sind umkehrbar. \\
Die Umkehrung von \(Z_i^\lb = Z_i^{\frac{1}{\lb}}\), die Umkehrung von \(Z_{i,j}^\lb = Z_{i,j}^{-\lb}\)

\subsubsection{Lemma 0.2}
Ein Gleichugnssystem G', welches aus einem Gleichungssytem G durch Zeilenoperationen hervorgeht besitzt die gleichen L\"osungen wie G.\\
Beweis:\\
F\"ur \(Z_I^\lb: \) betrachten wir nur die i-te Zeile.\\
\(a_{i,1}x_1 + \dots + a_{i,n}x_n = b_i\)\\
Nach \(Z_i^\lb\):
\(\lb a_{i,1}x_1 + \dots + \lb a_{i,n}x_n = \lb b_i\\\)
Diese besitzen eindeutig die selbe L\"osungen \(\xi_1, \dots, \xi_n\)\\
F\"ur \(Z_{i,j}^\lb\) ebenso.

\subsubsection{Satz 0.3}
Jedes lineare Gleichungssytem l\"asst sich durch Zeilenoperationen und Vertauschungen von Variablen (d.h. Vertauschung von Spalten) in Normalform bringen.\\
Beweis:\\
Wir beweisen dies mittels eines expliziten  Algorithmus (der Gau\ss{}=Jordan Elimination).\\
Aus praktischen Gruenden schreiben wir unser Gleichugnssystem als sogenannte erweiterte Koeffizientenmatrix.
\(
\begin{pmatrix}
a_11 & a_12 & \dots & a_1n &|& b_1\\
\vdots &&& &|& \\
a_m1 & a_m2 & \dots & a_mn &|& b_m\\
\end{pmatrix}
\)\\
Zunaechst vergewissern wir uns, dass wir durch nacheinander Anwendung von \(Z_{i,j}^1, Z_{j,i}^{-1}, Z_{i,j}^1\)und \( Z_{i}^{-1}\) die i-te und j-te Zeile vertauschen koennen.\\
Sei y die i-te Zeile, z die j-te Zeile.\\
\(
\vtwo{y}{z} \overset{Z_{i,j}^1}{\rightarrow}  \vtwo{y}{z+y} \overset{Z_{j, i}^{-1}}{\rightarrow}  \vtwo{-z}{z+y} \overset{Z_{i,j}^1}{\rightarrow}  \vtwo{-z}{y} \overset{Z_{i}^{-1}}{\rightarrow} \vtwo{z}{y}
\)\\
\\
\textbf{Algorithmus:}\\
\textbf{Schritt 1}: Falls alle Koeffizienten \(a_{i,j}\) Null sind, so ist die Matrix bereits in Normalform, und es ist nichts mehr zu tun.\\
Falls es einen von Null Verschiedenen Koeffizienten gibt, so k\"onnen wir diesen in die linke obere Ecke bringen (durch Spalten und Zeilenvertauschungen). Damit ist nun \(a_{1,1} \neq 0\). Nach \(Z_{1}^{\frac{1}{a_{1,1}}}\) gilt \(a_{1,1} = 1\). Nun wenden wir \(Z_{1,2}^{-a_{2,1}}, \dots, Z_{1,m}^{-a_{m,1}}\) und erhalten \(a_{2,1} = \dots = a_{m,1} = 0\).\\
Die Matrix hat nun die Form \(
\begin{pmatrix}
1 & a_{1, 2} & \dots & a_{1, n} &|& b_1\\
0\\
0\\
\vdots\\
0 & a_{m, 2} & \dots & a_{m, n} &|& b_m
\end{pmatrix}
\)\\
\textbf{Schritt 2}: Falls \(a_{i,j} = 0\) f\"ur \(2 \le i \le m\) und \(2 \le j \le n\), so ist die Matrix in Normalform f\"ur k=1 und wir sind fertig. Falls nicht, so existiert \(i \ge 2, j\ge 2\) mit \(a_{i,j} \neq 0\).\\
Wir vertauschen die i-te Zeile mit der zweiten Zeile, und die j-te Spalte mit der zweiten Spalte. Damit ist \(a_{2,2} \neq 0\). Nun wenden wir \(Z_{2}^{\frac{1}{a_{2,2}}}\) Damit ist \(a_{2,2} = 1\) . Danach wenden wir \(Z_{2,1}^{-a_{1,2}}, \dots, Z_{2,m}^{-a_{m,2}}\) an,\\ und erhalten die Form:
\(
\begin{pmatrix}
1 & 0 & a_{1, 3} & \dots & a_{1, n} &|& b_1\\
0 & 1 & a_{2, 3} & \dots & a_{1, n} &|& b_2\\
0 & 0\\
\vdots\\
0 & 0 & a_{m, 3} & \dots & a_{m, n} &|& b_m
\end{pmatrix}
\)\\
\(\vdots\)\\
Wir verwandeln Damit der Reihe nach die Spalten der Matrix in Spalten, in welchen nur der Diagonaleintrag von Null verschieden ist (dieser Eintrag ist gleich 1).\\
Das Verfahren terminiert, wenn die Matrix in Normalform ist, oder wenn \(min(n, m)\) Schritte vollzogen sind. Auch in diesem Fall ist die Matrix in Normalform.

\subsubsection{Korolar 0.4}
Sei A eine Matrix  mit m Zeilen und n Spalten. Weiter sei k der Rang einer Normalform von A (d.h. einer Matrix in Normalform, welche aus A durch Zeilenoperationen und Spaltenvertauschungen hervorgeht). Ein Gleichungssystem  mit Matrix A besitzt dann entweder keine L\"osung, oder ein (n-k) Parametriges L\"osungssytem. Es gilt \(k=n\) genau dann wenn jedes Gleichungssystem \(Ax=b\) h\"ochstens eine L\"osung besitzt und \(k=m\) genau dann wenn jedes Gleichungssytem \(Ax=b\) mindestens eine L\"osung besitzt.\\
\textbf{Beweis}: Folgt aus Lemma 0.2 und daraus, dass Zeilen / Spaltenoperationen die L\"osungsmenge (modulo Variablentausch) nicht \"andern.

\subsubsection{Korolar 0.5}
Ein homogenes Gleichungssystem mit weniger Gleichungen als Variablen hat mindestens eine nicht triviale L\"osung.\\
\textbf{Beweis}\\
Es gibt f\"ur homogene Gleichungssyteme immer die triviale L\"osung. Der Rang der Matrix des Gleichungssystems in Normalform sei k. Damit existiert ein (n-k) parametriges L\"osungssystem, aber \(k \le min(n, m) \le m \le (n-1)\). Somit existiert mindestens eine weitere L\"osung.\\

\subsubsection{Definition 0.6}
Eine Kollektion \(a_1, \dots, a_n\) von Vektoren in \(\R^m\) hei\ss{}t linear unabh\"angig, wenn sich keiner der Vektoren als Linearkombination der anderen Vektoren schreiben l\"asst.\\
\textbf{Bem:} Als Linearkombination von \(a_1, \dots, a_n\) bezeichnen  wir einen Ausdruck der Form \(\al_1a_1 + \al_2 a_2 + \dots + \al_n a_n = \sum_{j=1}^n \al_j a_j\) f\"ur \(\al_1, \dots, \al_n \in \R\)

\subsubsection{Lemma 0.7}
Vektoren \(a_1, \dots, a_n\) sind genau dann linear unabh\"angig, wenn f\"ur alle \(\xi_1, \dots, \xi_n \in\R\) gilt falls \(\xi_1a_1 + \dots + \xi_na_n = 0\)  dann gilt \(\xi_1 = \dots = \xi_n = 0\)\\
\textbf{Beweis}\\
1. Falls \(0 = \xi_1 a_1 + \dots + \xi_n a_n\), und oBdA. \(\xi_1 \neq 0\) so folgt \(a_1 = \sum_{j=2}^n -\frac{\xi_j}{\xi_1} a_j\). Somit habe ich \(a_1\) als Linearkombination von \(a_2, \dots, a_n\) geschrieben.\\
2. Falls aber oBdA. \(a_1 = \sum_{j=2}^n \lb_j a_j\) so gilt: \(0 = -a_1 = \sum{j=2}^n\), damit ist \(\xi_1\) (der erste Koeffizient) von Null verschieden.

\subsubsection{Lemma 0.8}
Es seien \(a_1, \dots, a_n \in \R^m\)  linear unabh\"angig und es gelte \(b = \lb_1a_1 + \dots + \lb_n a_n\), mit \(\lb_1, \dots, \lb_n \in \R\). Dann ist diese Linearkombination eindeutig.\\
\textbf{Beweis} Es sei auch \(b = \mu_1 a_1 + \dots + \mu_n a_n\). Fuer Eindeutigkeit ist nun zu zeigen, dass \(\mu_i = \lb_i, 1 \le i \le n\).\\
Wir ziehen die Gleichungen voneinander ab, und erhalten:\\
\(
b - b= (\lb_1 - \mu_1) a_1 + \dots + (\lb_n - \mu_n) a_n\\
\Leftrightarrow 0 = (\lb_1 - \mu_1) a_1 + \dots + (\lb_n - \mu_n) a_n\\
\)
Mit Lemma 0.7 folgt die Aussage.

\subsubsection{Satz 0.9}
Wenn man ein Gleichungssystem durch Zeilenoperationen und Spaltenvertauschungen auf Normalform bringt, so erh\"alt man immer denselben Rang.\\
\\
\textbf{Bemerkung} Man kann damit vom Rang eines Gleichungssystems (bzw. einer Matrix) sprechen, auch wenn dieses nicht in Normalform ist.\\
\textbf{Bemerkung} Ein einzelner Vektor a gilt als linear unabh\"angig, solange \(a \neq 0\). Die leere Kollektion von Vektoren (n=0) bezeichnen wir ebenfalls als linear unabh\"angig.\\
\\
Vor dem Beweis des Satzes 0.9 noch ein paar Feststellungen.\\
Die Tatsache, dass (\(\xi_1, \dots, \xi_n\) L\"osung eines linearen Gleichungssystems ist laesst sich als linaere Abhaengigkeit ausdruecken\\
\(\xi_1a_1 + \dots + \xi_n a_n = b\), wobei \(a_i\) eine Spalte der Matrix des Gleichungssystems ist.\\
Ist das Gleichungssystem in Normalform, so sind die ersten k Spaltenvektoren linear unabh\"angig. Die folgenden n-k Spaltenvektoren lassen sich aber als Linearkombination der ersten k darstellen, also\\
\(
\lb_{1,i}a_1 + \dots + \lb_{k,i}a_k = a_i \) fuer \( k < i \le n\\
\) mit \( \lb_{1,i} = a_{1,i}, \dots
\).
\\
Falls das Gleichungssystem l\"osbar ist, kann man dank \(\xi_1a_1 + \dots + \xi_n a_n = b\) auch b als solche Linearkombination schreiben.\\
Wegen Lemma 0.8 sind diese Linearkombinationen auch eindeutig.\\
\\
\textbf{Beweis von Satz 0.9}\\
Wir bemerken zun\"achst, dass Zeilenoperationen und Spaltenvertauschung die Anzahl linear unabh\"angiger Spaltenvektoren nicht \"andern.\\
Wir \"uberlegen uns nun, dass der Rang eines linearen Gleichungssystems nichts anderes als die maximale Anzahl linear unabh\"angiger Spaltenvektoren der Matrix ist.\\
\\
Die ersten k Spalten sind linear unabh\"angig, da die Matrix in Normalform.\\
Seien also \(a_{i_1}, \dots, a_{i_{k+1}}\) beliebige Spaltenvektoren der Matrix des Gleichungssystems. Nachdem in diesen Vektoren alle Eintr\"age ab dem k+1-ten Eintrag Null sind, hat das Gleichungssystem \\
\(
x_1a_{i_1} + \dots + x_{k+1}a_{i_{k+1}} = 0
\)\\
nur k m\"ogliche Gleichungen. (Die Zeilen k+1 bis m in diesem Gleichungssystem sind \(0=0\))\\
Nach Korolar 0.5 hat dieses homogene Gleichungssystem mir k Gleichungen und k+1 Unbekannten aber mindestens eine nicht triviale L\"osung. Die Vektoren \(a_{i_1}, \dots a_{i_{k+1}}\) sind somit nicht linear unabh\"angig.

\subsubsection{Korolar 0.10}
Wird ein Gleichungssystem \textit{nur} durch Zeilenoperationen (also ohne Variablentausch) auf Normalform gebracht, so ist die Matrix die man erh\"alt immer die gleiche. Falls das Gleichungssystem l\"osbar ist, so ist auch das erhaltene b immer das gleiche.

\subsection{Ein wenig euklidische Geometrie}

\subsubsection{Geraden und Ebenen}
\subsubsection{Def 0.11}
\begin{enumerate}
\item{
Sei v != 0 ein Vektor in \mRn. Mit \(\R v\) bezeichnen wir die Menge an Vektoren in \mRn der Form \(\R v = \{\lb v : \lb \in \R\}\)}
\item{
Sei \(a \in \Rn, v \in \Rn, v \neq 0\). Als (affine) Gerade bezeichnen wir die Menge der Vektoren der Form \(g = \{a + \lb v : \lb \in \R\} = a + \R v\)
}
\end{enumerate}
\textbf{Bemerkung}: Der Richtungsraum \(\R v\) einer Geraden g ist durch diese eindeutig bestimmt als Menge der Differenzen \(x - y\) aus Vektoren in g.\\

\subsubsection{Lemma 0.12}
Zwei Geraden \(a + \R v, b + \R w\) sind genau dann gleich, wenn gilt \(\R v = \R w\) und \(a - b \in \R v\).\\
\textbf{Beweis}\\
Sei also  \(x = a + \R v\), dh. \(x = a + \lb v\) f\"ur ein \(\lb \in \R\). Nach Annahme gilt \(\R v = \R w\). Damit existiert ein \(\mu \in \R\) mit \(\lb v = \mu w\) und somit \(x = a + \mu w\). Weiteres haben wir nach Annahme, dass \(a-b \in \R v\), also existiert ein \(\xi \in \R\) mit \(a - b = \xi w\), also \(x = a - (a - b) + \xi w + \mu w\) und sommit \(x = b + (\xi + \mu) w\).\\
Es ist also \(x \in b + \R w\).\\
Die Umkehrung, also die Behauptung, dass sich ein Punkt \(y \in b + \R w\) auch als Punkt in \(a + \R v\) schreiben l\"asst, folgt analog.\\

\subsubsection{Lemma 0.13}
Durch zwei verschiedene Punkte in \mRn geht genau eine Gerade.\\
Beweis: \"Ubung

\subsubsection{Definition 0.14}
Zwei Geraden hei\ss{}en parallel, wenn sie die gleichen Richtungsg\"aume haben.

\subsubsection{Definition 0.15}
Eine (affine) Ebene ist eine Menge der Form \(a + \R v + \R w\) f\"ur linear unabh\"angige Vektoren \(v, w\).\\
\textbf{Bemerkung}: Auch hier gilt, das der Raum \(\R v + \R w\) eindeutig bestimmt ist als Menge aller Differenzen von Punkten ind der Ebene.

\subsubsection{Lemma 0.16}
Zwei nicht-parallele Geraden, die in einer Ebene liegen, schneiden sich.\\
\textbf{Beweis}:\\
Es sei \(E = c + \R v_1 + \R v_2\), \(g_1 = a_1 + \R b_1\), \(g_2 = a_2 + \R b_2\) zwei Geraden in E.\\
Wir suchen \(\xi_1, \xi_1\), so dass \(a_1 + \xi_1 w_1 = a_2 + \xi_2 w_2\).\\
Nun schreiben wir \(a_i = c + \beta_{1,i} v_1 + \beta_{2,i} v_2\) und \(w_i = \al_{1,i} v_1 + \al_{2,i} v_2\) f\"ur \(i =1,2\).\\
Das f\"uhrt auf das Gleichungssystem\\
\(
\al_{1,1} \xi_1 - \al_{1,2} \xi_2 = - \beta_{1,1} + \beta_{1,2}\\
\al_{2,1} \xi_1 - \al_{2,2} \xi_2 = - \beta_{2,1} + \beta_{2,2}\\
\)\\
Nachdem \(g_1, g_2\) nicht parallel sind, sind \(w_1, w_2\) linear unabh\"angig. Damit sind aber die Spaltenvektoren der Matrix \(
\begin{pmatrix}
\al_{1,1} & -\al_{1,2}\\
\al_{2,1} & -\al_{2,2}
\end{pmatrix}
\) ebenfalls linear unabh\"angig. Damit besitzt das Gleichungssystem eine L\"osung (da \(k = m\)) nach Satz 0.9.

\subsubsection{Das Skalarprodukt}
Es seien \(a = (a_1, \dots, a_n), b = (b_1, \dots, b_n)\) zwei Vektoren in \mRn.\\

\subsubsection{Def 0.17}
Das Skalarprodukt von a und b ist definiert als \((a, b) = \sum_{j=1}^n a_j b_j\)
\end{document}