\documentclass{scrartcl}

\usepackage{hyperref}
\usepackage{amsfonts}
\usepackage{amssymb}
\usepackage{amsmath}
\usepackage[utf8]{inputenc}
\usepackage[german]{babel}
\usepackage{xstring}
\usepackage{tikz}

\title{Lineare Algebra I\\Mitschrieb}

% ease of use commands
\newcommand{\lb}{\lambda}
\newcommand{\mlb}{\(\lb\)}
\newcommand{\R}{\mathbb{R}}
\newcommand{\N}{\mathbb{N}}
\newcommand{\Z}{\mathbb{Z}}
\newcommand{\Q}{\mathbb{Q}}
\newcommand{\C}{\mathbb{C}}
\newcommand{\mR}{\(\mathbb{R}\)}
\newcommand{\mN}{\(\mathbb{N}\)}
\newcommand{\mZ}{\(\mathbb{Z}\)}
\newcommand{\mQ}{\(\mathbb{Q}\)}
\newcommand{\mC}{\(\mathbb{C}\)}
\newcommand{\Rn}{\mathbb{R}^n}
\newcommand{\mRn}{\(\mathbb{R}^n\)}
\newcommand{\al}{\alpha}
\newcommand{\be}{\beta}
\newcommand{\vtwo}[2]{\begin{pmatrix}#1 \\ #2 \end{pmatrix}}
\newcommand{\vthree}[3]{\begin{pmatrix}#1 \\ #2 \\ #3 \end{pmatrix}}
\newcommand{\ve}[1]{{\begin{pmatrix}#1 \end{pmatrix}}}
\renewcommand{\v}{\ve}

% increase line height
\renewcommand{\baselinestretch}{1.5}

% define table of theorems
\newcounter{tableoftheoremslinks}
\newcommand{\tableoftheorems}{\section*{S\"atze und Lemmata}}
%\newcommand{\addtotheorems}[2]{\stepcounter{tableoftheoremslinks} \label{tableoftheoremslinks} }
%\newcommand{\addtotheorems}[2]{%
%\stepcounter{tableoftheoremslinks}% Increase the link counter
%\label{theoremlink\thetableoftheoremslinks}#2% insert a label and the text
% \expandafter\edef\expandafter\tableoftheorems\expandafter%append to the %tableoftheorems command
% {\tableoftheorems % insert the command at its current state
% \hyperref[theoremlink\thetableoftheoremslinks]{#1}\\% insert a link to the label
% #2}}% insert the text
% dummy command
\newcommand{\dtheorem}[1]{#1}
\newcommand{\definition}[1]{#1}
\newcounter{test}
\newcommand{\test}{test}
\newcommand{\extest}[1]{\stepcounter{test} \label{a\thetest}\expandafter\edef\expandafter\test\expandafter{\test \hyperref[a\thetest]{test\thetest}}}


\begin{document}
\maketitle
\tableofcontents
\pagebreak

\section{Einf\"uhrung}
\begin{itemize}
\item{Das Wort Algebra stammt aus dem arabischen "al-jabr".}
\item{Allgemein ist Algebra die Lehre der mathematischen Symbole und deren Manipulation.}
\item{Lineare Algebra: Insbesondere lineare Gleichungen}
\end{itemize}

\subsection{Aufbau der Vorlesung}
\begin{enumerate}
\item{Lineare Gleichungssysteme und der n-dimensionale reellen Raum}
\item{Grundlegende Objekte}
\item{Gruppen, Ringe, K\"orper}
\item{Vektorr\"aume und lineare Abbildungen}
\item{Determinaten}
\item{Eigenwerte und Normalformen}
\end{enumerate}

\subsection{Beispiel: Der Google Pagerank}
Gegeben 4 Seiten, mit Verlinkungen zwischen den Seiten. Von einer nicht  verlinkten Seite wechselt man zuf\"llig auf eine andere Seite. Der user startet an einer zuf\"lligen Stelle und folgt von dort einem zuf\"alligen link auf eine andere Seite.Zus\"atzlich wird immer mit Wahrscheinlichkeit \((1-d), d \in [0, 1]\) auf eine beliebige Website gewechselt.\\
Die wichtigste Site ist nun die, auf welcher ein Benutzer sich mit der h\"ochsten Wahscheinlichkeit aufh\"alt.\\
\(
p(\delta_1) = \frac{1-d}{N} + d(\frac{p(\delta_2)}{1}, \frac{p(\delta_5)}{4})\\
p(\delta_2) = \frac{1-d}{N} + d(\frac{p(\delta_1)}{3}, \frac{p(\delta_5)}{4})\\
\vdots
\)\\
Zur Berechnung von \(p(\delta_j), j \in \{1..5\}\) gibt es Methoden aus der linearen Algebra.


\section{Lineare Gleichungssyteme und der n-dimensionale reelle Raum}
\begin{itemize}
\item{Descartes f\"uhrte ``Koordinaten'' ein in der Geometrie ein, also Zahlensysteme. Das f\"uhrte dazu, das man nun leichter rechnen kann.}
\item{Wir benutzen hier die reellen Zahlen (mit den \"ublichen Rechenregeln, also f\"ur die Addition :
\begin{itemize}
\item{\((x + y) + z = x + (y + z)\)}\\
\item{\(0 + x = x + 0 = x\)}
\item{Es gibt f\"ur jedes x ein y mit \(x + y = 0\), wir nennen dieses y das additiv inverse zu x (``-x'').}
\item{\(x + y = y + x\)}
\end{itemize}
Und f\"ur multiplikation:
\begin{itemize}
\item{\(\lambda (x + y) = \lambda x + \lambda y\)}
\item{\((\lambda + \mu) x = \lambda x + \mu x)\)}
\item{\(\lb(\rho\mu)=(\lb\rho)\mu\)}
\item{\(1x = x\)}
\end{itemize}
}
\item{Weiteres brauchen wir die nat\"urlichen Zahlen, die 1,2,3\dots}
\end{itemize}

\subsection{Der \(\R^n\)}
F\"ur gegebenes \(n \in \N\) definieren wir:\\
\(\R^n = \{x = (x_1, x_2, \dots, x_n): x_1, \dots, x_n \in \R\}\)\\
\((x_1, \dots, x_n)\) ist dabei ein geordnetes n-Tupel, die Reihenfolge beim Vergleich Elemente dieser Art ist wichtig.\\
F\"ur \(x, y, \in \R : x = y \Leftrightarrow x_1 = y_1, x_2 = y_2, \dots x_n, = y_n\)\\
Wir nennen diese n-Tupel auch Vektoren im \mRn.\\
Mit \(\R^0\) bezeichnen wir die Menge \(\{0\}\), welche nur das Nullelement enth\"alt.\\
Die Rechenregeln \"ubertragen sich nun von \mR. Wir schreiben\\
\(x + y = (x_1 + y_1, \dots, x_n + y_n)\) f\"ur \(x, y \in \Rn\) (Vektoradition)\\
\(\lb x = (\lb x_1, \dots, \lb x_n)\) (Skalarmultiplikation)

\subsection{Lineare Gleichungssyteme}
Eine lineare Gleichung ueber \mR ist ein Ausdruck der Form:\\
\(
\alpha_1x_1 + \alpha_2 x_2 + \dots \alpha_n x_n = \beta\)\\
Fuer reele Zahlen \(\beta, \alpha_1, \dots, \alpha_n \in \R\). Einen Vektor, \(\xi = \{\xi_1, \dots, \xi_n) \in \Rn\) nennen wir Loesung, wenn die reelen Zahleen  , \(\xi_1, \dots, \xi_n\) eingesetzt in \(x_1, \dots, x_n\) die Gleichung erf\"ullen.\\
Ein lineares Gleichungssystem G ist ein System \\
\(
a_11 x_1 + a_12 + x_2 \dots a_1n x_n = b_1\\
a_21 x_1 + a_22 + x_2 \dots a_2n x_n = b_1\\
\vdots\\
a_m1 x_1 + a_m2 + x_2 \dots a_mn x_n = b_1\\
\)\\
In Kurzform = \(\sum_{j=1}^n a_{i,j} x_j = b_i \forall i\in\{1,\dots,m\}\)\\
oder, noch k\"urzer, in Matrixschreibweise\\
\(Ax=b\)\\
Wobei \(A\) eine Matrix ist mit Eintr\"agen \(a_{i,j}\), wir schreiben\\
\(
A =
\begin{pmatrix}
a_{1, 1} & \dots & a_{1, n}\\
 & \vdots\\
a_{m, 1} & \dots & a_{m, n}
\end{pmatrix}
\)\\
\(Ax\) f\"ur \(x \in \Rn\) ist dann eine Kurzform f\"ur \(\sum_{i=1}^na_{ij}x_j\) mit einem Vektor \(x = (x_1, \dots, x_n) \in \Rn\). Das Ergebnis ist ein Vektor \(b = (b_1, \dots, b_m) \in \R^m\) f\"ur eine Matrix \(A\) mit m Zeilen und n Spalten.\\
Der Vektor b hei\ss{}t rechte Seite des linearen Gleichungssystems, A hei\ss{}t Koeffizienten Matrix des linearen Gleichungssystems. Eine Spalte / Zeile von A kann mit einem Vektor im \(\R^m\) bzw. im \(\R^n\) identifiziert werden. Wir sprechen von Spalten- / Zeilen Vektoren der Matrix A.\\
Eine Matrix mit m Zeilen und n Spalten nennen wir \(m\times n\) - Matrix. Fuer \(x  \in \Rn \), A eine \(m\times n\) - Matix und B eine lxm - Matrix gilt die Rechenregel \(BAx = B(Ax)\). Ein Gleichugnssystem \(Ax=b\) heisst homogen, falls b der Nullvektor \((0, \dots, 0)\) ist und quadratisch fuer \(m = n\) (eine quadratische Matrix A).\\
\subsubsection{Definition: Normalform}
Ein Gleichungssytstem \(Ax=b\) ist min Normalform, falss A die Gestalt \\
\(
\begin{pmatrix}
1 & 0 & a_{1,k+1}& \dots & a_{1, n}\\
0 & 1 & a_{m, k+1} & \dots & a_{m, n}\\
0 & 0 & 0 & 0 & 0\\
0 & 0 & 0 & 0 & 0\\
0 & 0 & 0 & 0 & 0\\
\end{pmatrix}
\) (fuer k=2)\\
fuer ein \(k \in \N_0\)\\
Beispiele:\\
\(
\begin{pmatrix}
1 & 0 & 3\\
0 & 1 & 4\\
0 & 0 & 0\\
0 & 0 & 0\\
\end{pmatrix}
\) Ist in Normalform f\"ur \(k = 2\).\\
\(
\begin{pmatrix}
1 & 0 & 0\\
0 & 1 & 0\\
0 & 0 & 1\\
\end{pmatrix}
\) Ist in Normalform f\"ur \(k = 3\).\\
\(
\begin{pmatrix}
0 & 0\\
0 & 0\\
\end{pmatrix}
\) Ist in Normalform f\"ur \(k = 0\).\\
k hei\ss{}t Rang der Matrix A (bzw. des Gleichungssytems). Es gilt \\
\(0 \le k \le min(m, n)\\ \)
Ein Gleichungssystem ist genau dann L\"osbar,  wenn gilt:\\
\(
b_{k+1} = b_{k+2} = \dots = b_m = 0
\)\\
In diesem Fall l\"asst sich eine L\"osung \(\xi \in \Rn\) bestimmen,  indem man \(\xi_{k+1}, \dots, \xi_n\) beliebig w\"ahlt, und danach \(\xi_i = b_i -  \sum_{j=k+1}^n a_{i,j} \xi_j \forall i\in\{1,\dots,n\}\) w\"ahlt.\\
Denn f\"ur Zeile \(i, i=k+1, \dots, n\) lautet das Gleichungssystem \(0x_1 + \dots + 0x_n = b_i = 0\) und \\
F\"ur Zeile \(i, i =1, \dots, k\) \\
\(a_{i, i} x_i + \sum_{j=k+1}^n a_{i,j} x_j = b_i\)\\
Beispiele:
\(
\begin{pmatrix}
1 & 0 & 3\\
0 & 1 & 4\\
0 & 0 & 0\\
0 & 0 & 0\\
\end{pmatrix} x = b =
\begin{pmatrix}
1\\1\\0\\0
\end{pmatrix}
\)\\
W\"ahle \(x_3 = 1\). Dann folgt daraus \(x_2 = -3\) und \(x_1 = -2\)\\
Wir sagen die L\"osungsmenge ist \\
\(\{(b_1 - \sum_{j=k+1}^na_{1j}\xi_j), \dots,  (b_k - \sum_j={k+1}^na_{kj}\xi_j), \xi_{k+1}, \dots, \xi_n : \xi_{k+1}, \dots, \xi_n \in \R\}\).\\
Wir nennen eine solche Mengen (n-k) parametrig.

\subsubsection{Lemma 0.1}
Sei A eine \(m\times n\) -Matrix mit Rang k. Dann gilt \(k=n\)  genau dann, wenn alle Gleichungssysteme mit A h\"ochstens eine L\"osung haben, und k = m, genau dann, wenn alle Gleichungssyteme mit A l\"osbar sind.\\
Beweis: klar aus der Darstellung.\\

\subsubsection{Zeilenoperationen}
Eine Zeilenoperation macht aus dem Gleichungssystem ein neues Gleichungssytem durch Multiplikation  der i-ten Zeile mit einer Zahl \(\lb \in \R \setminus 0\) oder durch addieren des \mlb -fachen der i-ten Zeile zur j-ten Zeile \((i \neq j)\). Wir bezeichnen diese Operationen mit \(Z_i^\lb\) bzw. \(Z_{i,j}^\lb\).\\
Bemerkung: Die Zeilenoperationen sind umkehrbar. \\
Die Umkehrung von \(Z_i^\lb = Z_i^{\frac{1}{\lb}}\), die Umkehrung von \(Z_{i,j}^\lb = Z_{i,j}^{-\lb}\)

\subsubsection{Lemma 0.2}
Ein Gleichugnssystem G', welches aus einem Gleichungssytem G durch Zeilenoperationen hervorgeht besitzt die gleichen L\"osungen wie G.\\
Beweis:\\
F\"ur \(Z_I^\lb: \) betrachten wir nur die i-te Zeile.\\
\(a_{i,1}x_1 + \dots + a_{i,n}x_n = b_i\)\\
Nach \(Z_i^\lb\):
\(\lb a_{i,1}x_1 + \dots + \lb a_{i,n}x_n = \lb b_i\\\)
Diese besitzen eindeutig die selbe L\"osungen \(\xi_1, \dots, \xi_n\)\\
F\"ur \(Z_{i,j}^\lb\) ebenso.

\subsubsection{Satz 0.3}
Jedes lineare Gleichungssytem l\"asst sich durch Zeilenoperationen und Vertauschungen von Variablen (d.h. Vertauschung von Spalten) in Normalform bringen.\\
Beweis:\\
Wir beweisen dies mittels eines expliziten  Algorithmus (der Gau\ss{}=Jordan Elimination).\\
Aus praktischen Gruenden schreiben wir unser Gleichugnssystem als sogenannte erweiterte Koeffizientenmatrix.
\(
\begin{pmatrix}
a_11 & a_12 & \dots & a_1n &|& b_1\\
\vdots &&& &|& \\
a_m1 & a_m2 & \dots & a_mn &|& b_m\\
\end{pmatrix}
\)\\
Zunaechst vergewissern wir uns, dass wir durch nacheinander Anwendung von \(Z_{i,j}^1, Z_{j,i}^{-1}, Z_{i,j}^1\)und \( Z_{i}^{-1}\) die i-te und j-te Zeile vertauschen koennen.\\
Sei y die i-te Zeile, z die j-te Zeile.\\
\(
\vtwo{y}{z} \overset{Z_{i,j}^1}{\rightarrow}  \vtwo{y}{z+y} \overset{Z_{j, i}^{-1}}{\rightarrow}  \vtwo{-z}{z+y} \overset{Z_{i,j}^1}{\rightarrow}  \vtwo{-z}{y} \overset{Z_{i}^{-1}}{\rightarrow} \vtwo{z}{y}
\)\\
\\
\textbf{Algorithmus:}\\
\textbf{Schritt 1}: Falls alle Koeffizienten \(a_{i,j}\) Null sind, so ist die Matrix bereits in Normalform, und es ist nichts mehr zu tun.\\
Falls es einen von Null Verschiedenen Koeffizienten gibt, so k\"onnen wir diesen in die linke obere Ecke bringen (durch Spalten und Zeilenvertauschungen). Damit ist nun \(a_{1,1} \neq 0\). Nach \(Z_{1}^{\frac{1}{a_{1,1}}}\) gilt \(a_{1,1} = 1\). Nun wenden wir \(Z_{1,2}^{-a_{2,1}}, \dots, Z_{1,m}^{-a_{m,1}}\) und erhalten \(a_{2,1} = \dots = a_{m,1} = 0\).\\
Die Matrix hat nun die Form \(
\begin{pmatrix}
1 & a_{1, 2} & \dots & a_{1, n} &|& b_1\\
0\\
0\\
\vdots\\
0 & a_{m, 2} & \dots & a_{m, n} &|& b_m
\end{pmatrix}
\)\\
\textbf{Schritt 2}: Falls \(a_{i,j} = 0\) f\"ur \(2 \le i \le m\) und \(2 \le j \le n\), so ist die Matrix in Normalform f\"ur k=1 und wir sind fertig. Falls nicht, so existiert \(i \ge 2, j\ge 2\) mit \(a_{i,j} \neq 0\).\\
Wir vertauschen die i-te Zeile mit der zweiten Zeile, und die j-te Spalte mit der zweiten Spalte. Damit ist \(a_{2,2} \neq 0\). Nun wenden wir \(Z_{2}^{\frac{1}{a_{2,2}}}\) Damit ist \(a_{2,2} = 1\) . Danach wenden wir \(Z_{2,1}^{-a_{1,2}}, \dots, Z_{2,m}^{-a_{m,2}}\) an,\\ und erhalten die Form:
\(
\begin{pmatrix}
1 & 0 & a_{1, 3} & \dots & a_{1, n} &|& b_1\\
0 & 1 & a_{2, 3} & \dots & a_{1, n} &|& b_2\\
0 & 0\\
\vdots\\
0 & 0 & a_{m, 3} & \dots & a_{m, n} &|& b_m
\end{pmatrix}
\)\\
\(\vdots\)\\
Wir verwandeln Damit der Reihe nach die Spalten der Matrix in Spalten, in welchen nur der Diagonaleintrag von Null verschieden ist (dieser Eintrag ist gleich 1).\\
Das Verfahren terminiert, wenn die Matrix in Normalform ist, oder wenn \(min(n, m)\) Schritte vollzogen sind. Auch in diesem Fall ist die Matrix in Normalform.

\subsubsection{Korolar 0.4}
Sei A eine Matrix  mit m Zeilen und n Spalten. Weiter sei k der Rang einer Normalform von A (d.h. einer Matrix in Normalform, welche aus A durch Zeilenoperationen und Spaltenvertauschungen hervorgeht). Ein Gleichungssystem  mit Matrix A besitzt dann entweder keine L\"osung, oder ein (n-k) Parametriges L\"osungssytem. Es gilt \(k=n\) genau dann wenn jedes Gleichungssystem \(Ax=b\) h\"ochstens eine L\"osung besitzt und \(k=m\) genau dann wenn jedes Gleichungssytem \(Ax=b\) mindestens eine L\"osung besitzt.\\
\textbf{Beweis}: Folgt aus Lemma 0.2 und daraus, dass Zeilen / Spaltenoperationen die L\"osungsmenge (modulo Variablentausch) nicht \"andern.

\subsubsection{Korolar 0.5}
Ein homogenes Gleichungssystem mit weniger Gleichungen als Variablen hat mindestens eine nicht triviale L\"osung.\\
\textbf{Beweis}\\
Es gibt f\"ur homogene Gleichungssyteme immer die triviale L\"osung. Der Rang der Matrix des Gleichungssystems in Normalform sei k. Damit existiert ein (n-k) parametriges L\"osungssystem, aber \(k \le min(n, m) \le m \le (n-1)\). Somit existiert mindestens eine weitere L\"osung.\\

\subsubsection{Definition 0.6}
Eine Kollektion \(a_1, \dots, a_n\) von Vektoren in \(\R^m\) hei\ss{}t linear unabh\"angig, wenn sich keiner der Vektoren als Linearkombination der anderen Vektoren schreiben l\"asst.\\
\textbf{Bem:} Als Linearkombination von \(a_1, \dots, a_n\) bezeichnen  wir einen Ausdruck der Form \(\al_1a_1 + \al_2 a_2 + \dots + \al_n a_n = \sum_{j=1}^n \al_j a_j\) f\"ur \(\al_1, \dots, \al_n \in \R\)

\subsubsection{Lemma 0.7}
Vektoren \(a_1, \dots, a_n\) sind genau dann linear unabh\"angig, wenn f\"ur alle \(\xi_1, \dots, \xi_n \in\R\) gilt falls \(\xi_1a_1 + \dots + \xi_na_n = 0\)  dann gilt \(\xi_1 = \dots = \xi_n = 0\)\\
\textbf{Beweis}\\
1. Falls \(0 = \xi_1 a_1 + \dots + \xi_n a_n\), und oBdA. \(\xi_1 \neq 0\) so folgt \(a_1 = \sum_{j=2}^n -\frac{\xi_j}{\xi_1} a_j\). Somit habe ich \(a_1\) als Linearkombination von \(a_2, \dots, a_n\) geschrieben.\\
2. Falls aber oBdA. \(a_1 = \sum_{j=2}^n \lb_j a_j\) so gilt: \(0 = -a_1 = \sum{j=2}^n\), damit ist \(\xi_1\) (der erste Koeffizient) von Null verschieden.

\subsubsection{Lemma 0.8}
Es seien \(a_1, \dots, a_n \in \R^m\)  linear unabh\"angig und es gelte \(b = \lb_1a_1 + \dots + \lb_n a_n\), mit \(\lb_1, \dots, \lb_n \in \R\). Dann ist diese Linearkombination eindeutig.\\
\textbf{Beweis} Es sei auch \(b = \mu_1 a_1 + \dots + \mu_n a_n\). Fuer Eindeutigkeit ist nun zu zeigen, dass \(\mu_i = \lb_i, 1 \le i \le n\).\\
Wir ziehen die Gleichungen voneinander ab, und erhalten:\\
\(
b - b= (\lb_1 - \mu_1) a_1 + \dots + (\lb_n - \mu_n) a_n\\
\Leftrightarrow 0 = (\lb_1 - \mu_1) a_1 + \dots + (\lb_n - \mu_n) a_n\\
\)
Mit Lemma 0.7 folgt die Aussage.

\subsubsection{Satz 0.9}
Wenn man ein Gleichungssystem durch Zeilenoperationen und Spaltenvertauschungen auf Normalform bringt, so erh\"alt man immer denselben Rang.\\
\\
\textbf{Bemerkung} Man kann damit vom Rang eines Gleichungssystems (bzw. einer Matrix) sprechen, auch wenn dieses nicht in Normalform ist.\\
\textbf{Bemerkung} Ein einzelner Vektor a gilt als linear unabh\"angig, solange \(a \neq 0\). Die leere Kollektion von Vektoren (n=0) bezeichnen wir ebenfalls als linear unabh\"angig.\\
\\
Vor dem Beweis des Satzes 0.9 noch ein paar Feststellungen.\\
Die Tatsache, dass (\(\xi_1, \dots, \xi_n\) L\"osung eines linearen Gleichungssystems ist laesst sich als linaere Abhaengigkeit ausdruecken\\
\(\xi_1a_1 + \dots + \xi_n a_n = b\), wobei \(a_i\) eine Spalte der Matrix des Gleichungssystems ist.\\
Ist das Gleichungssystem in Normalform, so sind die ersten k Spaltenvektoren linear unabh\"angig. Die folgenden n-k Spaltenvektoren lassen sich aber als Linearkombination der ersten k darstellen, also\\
\(
\lb_{1,i}a_1 + \dots + \lb_{k,i}a_k = a_i \) fuer \( k < i \le n\\
\) mit \( \lb_{1,i} = a_{1,i}, \dots
\).
\\
Falls das Gleichungssystem l\"osbar ist, kann man dank \(\xi_1a_1 + \dots + \xi_n a_n = b\) auch b als solche Linearkombination schreiben.\\
Wegen Lemma 0.8 sind diese Linearkombinationen auch eindeutig.\\
\\
\textbf{Beweis von Satz 0.9}\\
Wir bemerken zun\"achst, dass Zeilenoperationen und Spaltenvertauschung die Anzahl linear unabh\"angiger Spaltenvektoren nicht \"andern.\\
Wir \"uberlegen uns nun, dass der Rang eines linearen Gleichungssystems nichts anderes als die maximale Anzahl linear unabh\"angiger Spaltenvektoren der Matrix ist.\\
\\
Die ersten k Spalten sind linear unabh\"angig, da die Matrix in Normalform.\\
Seien also \(a_{i_1}, \dots, a_{i_{k+1}}\) beliebige Spaltenvektoren der Matrix des Gleichungssystems. Nachdem in diesen Vektoren alle Eintr\"age ab dem k+1-ten Eintrag Null sind, hat das Gleichungssystem \\
\(
x_1a_{i_1} + \dots + x_{k+1}a_{i_{k+1}} = 0
\)\\
nur k m\"ogliche Gleichungen. (Die Zeilen k+1 bis m in diesem Gleichungssystem sind \(0=0\))\\
Nach Korolar 0.5 hat dieses homogene Gleichungssystem mir k Gleichungen und k+1 Unbekannten aber mindestens eine nicht triviale L\"osung. Die Vektoren \(a_{i_1}, \dots a_{i_{k+1}}\) sind somit nicht linear unabh\"angig.

\subsubsection{Korolar 0.10}
Wird ein Gleichungssystem \textit{nur} durch Zeilenoperationen (also ohne Variablentausch) auf Normalform gebracht, so ist die Matrix die man erh\"alt immer die gleiche. Falls das Gleichungssystem l\"osbar ist, so ist auch das erhaltene b immer das gleiche.

\subsection{Ein wenig euklidische Geometrie}

\subsubsection{Geraden und Ebenen}
\subsubsection{Def 0.11}
\begin{enumerate}
\item{
Sei v != 0 ein Vektor in \mRn. Mit \(\R v\) bezeichnen wir die Menge an Vektoren in \mRn der Form \(\R v = \{\lb v : \lb \in \R\}\)}
\item{
Sei \(a \in \Rn, v \in \Rn, v \neq 0\). Als (affine) Gerade bezeichnen wir die Menge der Vektoren der Form \(g = \{a + \lb v : \lb \in \R\} = a + \R v\)
}
\end{enumerate}
\textbf{Bemerkung}: Der Richtungsraum \(\R v\) einer Geraden g ist durch diese eindeutig bestimmt als Menge der Differenzen \(x - y\) aus Vektoren in g.\\

\subsubsection{Lemma 0.12}
Zwei Geraden \(a + \R v, b + \R w\) sind genau dann gleich, wenn gilt \(\R v = \R w\) und \(a - b \in \R v\).\\
\textbf{Beweis}\\
Sei also  \(x = a + \R v\), dh. \(x = a + \lb v\) f\"ur ein \(\lb \in \R\). Nach Annahme gilt \(\R v = \R w\). Damit existiert ein \(\mu \in \R\) mit \(\lb v = \mu w\) und somit \(x = a + \mu w\). Weiteres haben wir nach Annahme, dass \(a-b \in \R v\), also existiert ein \(\xi \in \R\) mit \(a - b = \xi w\), also \(x = a - (a - b) + \xi w + \mu w\) und sommit \(x = b + (\xi + \mu) w\).\\
Es ist also \(x \in b + \R w\).\\
Die Umkehrung, also die Behauptung, dass sich ein Punkt \(y \in b + \R w\) auch als Punkt in \(a + \R v\) schreiben l\"asst, folgt analog.\\

\subsubsection{Lemma 0.13}
Durch zwei verschiedene Punkte in \mRn geht genau eine Gerade.\\
Beweis: \"Ubung

\subsubsection{Definition 0.14}
Zwei Geraden hei\ss{}en parallel, wenn sie die gleichen Richtungsg\"aume haben.

\subsubsection{Definition 0.15}
Eine (affine) Ebene ist eine Menge der Form \(a + \R v + \R w\) f\"ur linear unabh\"angige Vektoren \(v, w\).\\
\textbf{Bemerkung}: Auch hier gilt, das der Raum \(\R v + \R w\) eindeutig bestimmt ist als Menge aller Differenzen von Punkten in der Ebene.

\subsubsection{Lemma 0.16}
Zwei nicht-parallele Geraden, die in einer Ebene liegen, schneiden sich.\\
\textbf{Beweis}:\\
Es sei \(E = c + \R v_1 + \R v_2\), \(g_1 = a_1 + \R b_1\), \(g_2 = a_2 + \R b_2\) zwei Geraden in E.\\
Wir suchen \(\xi_1, \xi_1\), so dass \(a_1 + \xi_1 w_1 = a_2 + \xi_2 w_2\).\\
Nun schreiben wir \(a_i = c + \beta_{1,i} v_1 + \beta_{2,i} v_2\) und \(w_i = \al_{1,i} v_1 + \al_{2,i} v_2\) f\"ur \(i =1,2\).\\
Das f\"uhrt auf das Gleichungssystem\\
\(
\al_{1,1} \xi_1 - \al_{1,2} \xi_2 = - \beta_{1,1} + \beta_{1,2}\\
\al_{2,1} \xi_1 - \al_{2,2} \xi_2 = - \beta_{2,1} + \beta_{2,2}\\
\)\\
Nachdem \(g_1, g_2\) nicht parallel sind, sind \(w_1, w_2\) linear unabh\"angig. Damit sind aber die Spaltenvektoren der Matrix \(
\begin{pmatrix}
\al_{1,1} & -\al_{1,2}\\
\al_{2,1} & -\al_{2,2}
\end{pmatrix}
\) ebenfalls linear unabh\"angig. Damit besitzt das Gleichungssystem eine L\"osung (da \(k = m\)) nach Satz 0.9.

\subsubsection{Das Skalarprodukt}
Es seien \(a = (a_1, \dots, a_n), b = (b_1, \dots, b_n)\) zwei Vektoren in \mRn.\\

\subsubsection{Def 0.17}
Das Skalarprodukt von a und b ist definiert als \((a, b) = \sum_{j=1}^n a_j b_j\).

\subsubsection{Lemma 0.18}
Das Skalarprodukt zweier Vektoren a und b in \mRn ist eine sogennante symmetrische, positiv definite Bilinearform, das hei\ss{}t.
\begin{enumerate}
\item{\((a, b) = (b, a)\) (symmetrisch)}
\item{\((a + b, c) = (a, c) + (b, c)\) (linear)}
\item{\(\lb a, b = \lb(a, b)\) (linear)}
\item{\((a, a) \ge 0\) (positiv definit)}
\item{\((a, a) = 0\) genau dann, wenn \(a=0\)}
\end{enumerate}
f\"ur alle Vektoren \(a, b, c \in \Rn\), alle \(\lb \in \R\).\\
\textbf{Bemerkung}: aus 1 und 2 folgt \((a, b+c = (a,b) + (a,c)\) und \((a, \lb b) = \lb (a, b)\) (bilinearit\"at).\\
\textbf{Beweis}: 1, 2, 3 sind klar aus der Definition.\\
4 und 5 folgen daraus, dass \((a, a) = a_1^2, \dots, a_n^2\).\\

\subsubsection{Def. 0.19 Norm eines Vektors}
Die Norm (oder L\"ange) von a ist \(\sqrt{(a, a)} = ||a||\).

\subsubsection{Def. 0.20 Winkel zweier Vektoren}
\begin{enumerate}
\item {
Der Winnkel \(\al\) zwischen zwei Vektoren \(a, b \neq 0\) ist definiert durch \(0 \le \al \le \pi\) und \(cos(\al) = \frac{|(a,b)|}{||a||\cdot ||b||}\).}
% TODO chang to \frac{(a,b)}{||a||\cdot ||b||}, so that angle is actually between 0 and pi?
\item{
Zwei Vektoren \(a, b \in \R^n\) hei\ss{}en orthogonal, falls gilt \((a, b) = 0\).
}
\end{enumerate}

\subsubsection{Lemma 0.21 Cauchy-Schwarz'sche Ungleichung}
Es gilt \(|(a, b)| \le ||a||||b||\).\\
\textbf{Beweis}:\\
Es gilt f\"ur jedes beliebiges \(\lb \in \R\):\\
\(0 \le (a + \lb b, a + \lb b) = (a, a) + 2 (\lb a, b) + \lb^2 (b, b)\)\\
F\"ur \(\lb = -\frac{(a,b)}{(b, b)}\) ergibt sich\\
\(0 \le (a, a) - 2 \frac{(a, b)^2}{(b, b)} + \frac{(a, b)^2}{(b, b)}\)\\
F\"ur \(b = 0\) ist die Aussage des Lemmas klar. Angenommen \(b \neq 0\). Es folgt:\\
\((a,b)^2 \le (a, a)(b,b)\)\\
\textbf{Bemerkung}: Falls a und b linear unabh\"angig sind so folgt \(|(a,b)| < ||a||||b||\), denn dann ist \(a + \lb b \neq 0\) (f\"ur jedes \(\lb \in \R\)) und die Ungleichung ist strikt (d.h. mit "\(<\)").

\subsubsection{Lemma 0.22 Dreiecksungleichung}
Es gilt \(||a+b|| \le ||a|| + ||b||\).\\
\textbf{Beweis:}\\
\( ||a+b||^2 = (a+b, a+b)= ||a||^2 + 2(a,b) + ||b||^2 \le ||a||^2 + 2 ||a||||b|| + ||b||^2 = (||a|| + ||b||)^2\)

\subsubsection{Korollar 0.23 \(||x-y||\) ist eine Metrik}
Der \mRn mit dem Abstand \(d(x, y) = ||x - y||\) ist ein sogennanter metrischer Raum. D.h.
\begin{enumerate}
\item{
\(d(x, y) \ge 0\)}
\item{\(d(x, y) = 0 \Leftrightarrow x = y\)}
\item{\(d(x, y) = d(y, x)\)}
\item{\(d(x, z) \le d(x,y) + d(y,z)\)}
\end{enumerate}
f\"ur alle x, y, z in \mRn.\\
Wir nennen d einen Abstand.

\section{Grundlegende Objekte}

\subsection{Elementare Aussagenlogik}
Aussagen (in der Mathematik) sind sprachliche Gebilde, welche entweder wahr (w) oder falsch (f) sind.\\
Darstellung mittels Wahrheitstabelle:\\
Beispiele:
\begin{tabular}{l | l}
Aussage\\
\hline
A: es sind am 2.11.2017 mehr als f\"unf Personen im H\"orsaal Rundbau. & w\\
\hline
B = Der Dozent der LA in FR im WS 17/18 hei\ss{}t Peter & f
\end{tabular}

\subsubsection{Definition 1.1 Logische Operatoren}
A, B seien Aussagen.
\begin{enumerate}
\item{"\(\neg A\)", oder ''nicht A'' ist die Negation von A\\
\begin{tabular}{l | l}
A & \(\neg A\)\\
w & f\\
f & w
\end{tabular}
}
\item{Junktoren\\
\(A \lor B\), ''A oder B'' ist wahr, wenn mindestens eine der Aussagen wahr A, B ist.\\
\(A \land B\), ''A und B'' ist wahr, wenn beide wahr sind.\\
\begin{tabular}{l | l | l | l}
A & b & \(A \lor B\) & \(A\land B\)\\
\hline
w & w & w & w\\
f & w & w & f\\
w & f & w & f\\
f & f & f & f
\end{tabular}
}
\item{Implikationen\\
\(A \Rightarrow B\) ist wahr, wenn A die Aussage B impliziert.\\
\(A \Leftrightarrow B\)  ist wahr, wenn A genau dann wahr ist, wenn B wahr ist.\\
\begin{tabular}{l | l | l | l}
A & B & \(A \Rightarrow B\) & \(A \Leftrightarrow B\)\\
\hline
w & w & w & w\\
f & w & w & f\\
w & f & f & f\\
f & f & w & w
\end{tabular}
}
\end{enumerate}
\textbf{Beispiel} Sei G ein lineares Gleichungssystem mit m Zeilen, n Spalten und Grad k. Dann gilt \\
\begin{tabular}{l c l}
\(k = n\) & \(\Rightarrow\) & L\"osung immer eindeutig.\\
\(A\) & \(\Rightarrow\) &  \(B\)\\
\end{tabular}\\
Um die Aussage \(A \Rightarrow B\) zu zeigen, k\"onnen wir annehmen, das A richtig ist und m\"ussen folgern, das B auch richtig ist.\\
\\
\textbf{Bemerkung (De Morgan)}
\begin{enumerate}
\item{\((\neg A \lor \neg B) = \neg (A \land B)\)}
\item{\((\neg A \land \neg B\) = \(\neg (A \lor B)\)}
\end{enumerate}

\subsection{Mengen und Abbildungen}
Problem: Der Begriff der Menge ist sehr schwer zu definieren. (Die Menge aller Mengen die sich nicht selbst enthalten, ist zwar naiv eine Menge, macht aber keinen Sinn, da die Definition dieser Menge zum Widerspruch gef\"uhrt werden kann. Objekte wie dieses machen die Definition schwer.)\\
Endliche Mengen kann man durch Auflistung aller Elemente angeben.\\
z.B. \(X = \{x_1, x_2, x_3\}\)\\
\(x_1, x_2, x_3\) hei\ss{}en dann Elemente von X und wir scheiben \(x_1 \in X\).\\
Reihenfolge der Elemente und Mehrfachauflistung sind nicht relevant. Die M\"achtigkeit einer Menge ist die Anzahl paarweise verschiedener Elemente.\\
\(\{1, 2, 2, 3\}\) hat M\"achtigkeit 3.\\
Die leere Menge \(\{\}\) oder \(\emptyset\) enth\"alt kein Element.

\subsubsection{Definition 1.2 Teilmengen und Gleichheit}
\begin{enumerate}
\item{Eine Menge Y hei\ss{}t Teilmenge von X, wenn aus \(x \in y\) immer folgt \(x \in X\). Wir schreiben \(Y \subset X\).}
\item{Wir sagen \(X=Y\) genau dann, wenn \(X \subset Y\) und \(Y \subset X\)\\
(d.h. zwei Mengen sind gleich, wenn sie die gleichen Elemente enthalten. (''Extensenalit\"astsprinzip''?))}
\end{enumerate}
\textbf{Bemerkungen}
\begin{enumerate}
\item{\(\emptyset \subset M\), f\"ur jede Menge M}
\item{\(M \subset M\), f\"ur jede Menge M}
\item{Wenn gilt \(M \subset N\), aber nicht \(M = N\), dann hei\ss{}t M ''echte Teilmenge'' von N, wir schreiben dann \(M \subsetneq N\) (Die ISO vorschrift sieht hier \(\subset\) f\"ur ''echte Teilmenge'' und \(\subseteq\) f\"ur ''Teilmenge'' vor, dies wird jedoch selten bentuzt.)}
\end{enumerate}

\subsubsection{Die Nat\"urliche Zahlen}
Die einfachste unendliche Menge ist die, der nat\"urlichen Zahlen\\
\(\N = \{1, 2, 3, \dots\}\), deren Existenz wir annehmen, zusammen mit den \"ublichen Rechenregeln.\\
Die nat\"urlichen Zahlen gen\"ugen dem Prinzip der vollst\"andigen Induktion.\\
Sei \(M \subset \N\) und es gelte:
\begin{enumerate}
\item{\(1 \in M\)}
\item{falls \(m \in M\), so ist auch \(n + 1 \in M\)}
\end{enumerate}
Dann gilt \(M = \N\).\\
\\
Durch Erweiterung von Zahlbereichen k\"onnen wir aus \mN auch die ganzen Zahlen \(\mathbb{Z}\), die rationalen Zahlen \(\mathbb{Q}\) sowie die reellen Zahlen \mR konstruieren.\\
(ebenso die komplexen Zahlen \(\mathbb{C}\))\\
\textbf{Bemerkung}\\
Es gilt \(\N \subset \mathbb{Z} \subset \mathbb{Q} \subset \mathbb{R} \subset \mathbb{C}\)\\

\subsubsection{Teilmengen mit Eigenschaften}
Aus einer Menge k\"onnen wir Teilmengen Ausw\"ahlen, welche durch bestimme Eigenschaften charakterisiert werden. Wir schreiben\\
\(X' = \{x \in X : x \) hat Eigenschaft E\(\}\) (aucch \(\{x \in X | x \) hat Eigenschaft\(\}\) ist verbreitet)

\subsubsection{Definition 1.3 Mengenoperationen }
Sind X, Y Mengen, so k\"onnen wir bilden:
\begin{enumerate}
\item{Die Vereinigung \(X \cup Y\), ist die Menge aller Elemente, welche in X sind oder welche in Y sind.}
\item{Der Schnitt \(X \cap Y = \{x \in X : x \in Y\}\), ist die Menge aller Elemente, die sowohl in X als auch in Y sind.}
\item{F\"ur \(Y \subset X\) schreiben wir \(X \setminus Y\) sprich ''X ohne Y'' f\"ur die Menge \(\{x \in X : x \not\in Y\}\)}
\item{Das ''kartesische Produkt'' \(X \times Y\) ist die Menge aller geordneten Tupel \(\{(x, y) : x\in X, y\in Y\}\) }
\end{enumerate}
\textbf{Beispiele}
\begin{enumerate}
\item{\(\{1, 2, 4\} \cap \{2, 3\} = \{2\}\)}
\item{\(\R \times \R = \R^2\)}
\item{Die Elemente der Menge \(\{1, \{1\}, 2\}\) sind genau \(1, \{1\}, 2\) }
\end{enumerate}

\subsubsection{Definition 1.4 Abbildungen}
Seien X, Y Mengen. Als Abbildung von X nach Y bezeichnen wir eine Vorschrift f, welche jedem Element x in X genau ein Element y in Y zuordnet. Wir schreiben\\
\(f : X \rightarrow Y\), \(x \mapsto f(x)\)

\subsubsection{Definition 1.5 Gleichheit von Abbildungen}
Zwei Abbildungen \(f : X \rightarrow Y, g : X \rightarrow Y\) hei\ss{}en gleich, wenn f\"ur alle \(x \in X\) gilt \(f(x) = g(x)\).

\subsubsection{Definition 1.6 Bild und Urbild}
Sei \(f: X\rightarrow Y, M \subset X, N \subset Y\)
\begin{enumerate}
\item{Wir schreiben   \(f(M) = \{y \in Y : \) es existiert \(x \in M\) mit \(f(x) = y \} \subset Y\) Bild von M}
\item{\(f^{-1}(N) = \{x \in X : f(x) \in N\} \subset X\) Urbild von N}
\end{enumerate}
\textbf{Beispiel}:
\begin{enumerate}
\item{\(X = \{1, 2, 3\}\\
Y = \{3, 4, 5, 6\}\\
f(1) = 4, f(2) = 5, f(3) = 5\\
M = \{1, 2\} \subset X\\
f(M) = \{4, 5\} \subset Y\\
f(\emptyset) = \emptyset \subset Y\\
f(X) = \{4, 5\}\\
N = \{3, 4, 5\}\\
f^{-1}(N) = \{1, 2, 3\}\\
f^{-1}(\emptyset) = \emptyset\\
f^{-1}(\{6\}) = \emptyset\\
f^{-1}(\{5\}) = \{2, 3\}\\
\)
}
\item{
\(X = \R., Y = \R\\
f : X \rightarrow Y, x \mapsto f(x) =  x^2\\
f([1,2]) = [1, 4] \subset Y\\
f^{-1}(\{0\}) = \{0\}\\
f^{-1}(\{1\}) = \{-1, 1\}\\
f^{-1}(\{-1\}) = \emptyset\\
\)
}
\end{enumerate}
\textbf{Achtung}: \(f^{-1}(N)\) ist nur definiert f\"ur Mengen \(N \subset Y\). Insebesundere ist \(f^{-1}\) (zumindest jetzt) keine Abbildung von Y nach X.

\subsubsection{Definition 1.7 Einschr\"ankung von Funktionen}
Es sei \(f : X \to Y\) eine Abbildung, \(M \subset X \).\\
Die Einschr\"ankung von f auf M ist die Abbildung \(f|_M = M \to Y, x \mapsto f(x)\)\\
\textbf{Bemerkung}\\
Der Unterschied zu f ist nur der eingeschr\"ankte Definitionsbereich.\\
\(f  :\R \to \R, x \mapsto f(x) = x^2\\
M = R^+_0 = \{x \in \R : x \ge 0\}\\
(f|_M)^{-1}(\{1\}) = \{1\}
\)

\subsubsection{Def 1.8 Injektiv und surjektiv}
Es sei \(f : X \to Y\) eine Abbildung.
\begin{enumerate}
\item{f hei\ss{}t injektiv, falls gilt\\
\((x, x'\in X, f(x) = f(x') \Rightarrow x = x')\)}
\item{
f hei\ss{}t surjektiv, falls gilt\\
\(f(X) = Y\)
}
\item {
 f hei\ss{}t bijektiv, falls f injektiv und surjektiv ist.
}
\end{enumerate}
\textbf{Beispiel}\\
\(f : \R \to \R, x \mapsto f(x) = x^2\)\\
ist nicht injektiv, da \(f(-1) = f(1), 1 \neq -1\). f ist auch nicht surjektiv, da \(f(x) \ge 0\).\\
\(f|_{\R^+_0} : \R^+_0 \to \R\) ist injektiv, aber nicht surjektiv\\
\(f|_{\R^+_0} : \R^+_0 \to \R^+_0\) ist injektiv, und surjektiv, also bijektiv\\

\subsubsection{Definition 1.9 \(f^{-1}\)}
Es sei \(f : X \to Y\) bijektiv. Wir schreiben dann \(f^{-1} : Y \to X, f^{-1}(y) = x\) mit dem eindeutig definierten \(x \in X\), sodass gilt \(f(x) = y\).\\
\textbf{Bemerkung}\\
Die Sinnhaftigkeit der Definition 1.9 folgt sofort aus der Definition von bijektivit\"at.

\subsubsection{Satz 1.10}
Sei X eine endliche Menge, so sind f\"ur \(f : X \to X\) \"aquivalent:
\begin{enumerate}
\item{f ist injektiv}
\item{f ist surjektiv}
\item{f ist bijektiv}
\end{enumerate}
\textbf{Bemerkung} F\"ur nicht endliche Mengen haben wir einfache Gegenbeispiele:\\
\(f : \N \to \N , x \mapsto f(x) = 2x\)\\
\textbf{Beweis}\\
X ist eine endliche Menge, wir schreiben \(X = \{x_1, \dots, x_n\}\) mit paarweise verschiedenen \(x_j\).
\begin{enumerate}
\item{Wir zeigen zun\"achst \(1. \Rightarrow 2.\). Zu zeigen ist also falls f injektiv ist, so ist f auch surjektiv. Dies wird impliziert durch die Aussage ''Ist f \textit{nicht} surjektiv, so ist f auch \textit{nicht} injektiv'', welche wir zeigen.\\
Sei f also nicht surjektiv. Also \(f(X) \neq X\). Damit besteht \(f(X)\) aus \(m < n\) Elementen. Verteilt man aber n Elemente in \(m < n\) Schubladen, so muss eien Schublade existieren, in der mehr als ein Element ist. Damit kann f nicht injektiv sein (es existiert \(x \neq x'\) mit \(f(x') = f(x)\)).}
\item{\(2. \Rightarrow 1.\). Sei f also nicht injektiv, dann existieren nach Definition \(x, x' \in X, x' \neq x\) aber \(f(x) = f(x')\). Damit kann aber \(f(X)\) h\"ochstens n-1 Elemente enthalten und f ist auch nicht surjektiv.}
\item{\(3. \Rightarrow 1.\) trivial nach der Definition der Bijektivit\"at}
\item{\(3. \Rightarrow 2.\) ebenso}
\item{\(1. \Rightarrow 3.\) Aus Injektivit\"at folgt bereits Surjektivit\"at und damit auch Bijektivit\"at.}
\item{\(2. \Rightarrow 3.\) Aus Surjektivit\"at folgt bereits Injektivit\"at und damit auch Bijektivit\"at.}
\end{enumerate}

\subsubsection{Definition 1.11 Komposition von Abbildungen}
Es seien X, Y, Z Mengen, \(f: X \to Y, g : Y \to Z\) Abbildungen.\\
Dann definiert \(g \circ f : X \to Z, x \mapsto g(f(x)) = (g \circ f)(x)\) die Komposition von Abbildungen.\\
\\
\textbf{Bemerkung}\\
Es gilt Assoziativit\"at: \((h \circ g) \circ f = h \circ (g \circ f)\) f\"ur \(f : X \to Y, g : Y \to Z , h : Z \to A\)\\
aber \textit{nicht} Kommutativit\"at, d.h. im Allgemeinen gilt nicht  \(f\circ g = g \circ f\) f\"ur \(f : X \to X, g : X \to X\), denn \\
\(f : \R \to R, f(x) = x + 1\\
g : \R \to \R, f(x) = x^2\)\\
ist ein Gegenbeispiel, denn im Allgemeinen gilt \textit{nicht}, dass \((x + 1)^2 = x^2 + 1\).

\subsubsection{Definition 1.12 Identit\"at}
Mit  \(Id_X : X \to X\) bezeichnen wir die identische Abbildung \(x \mapsto x\)

\subsubsection{Lemma 1.13 Identit\"at und Surjektivit\"at bzw. Injektivit\"at}
Es sei \(f : X \to Y\) eine Abbildung, \(X, Y \neq \emptyset\). Dann gilt:
\begin{enumerate}
\item{f ist genau dann injektiv, wenn eine Abbildung \(g : Y \to X\) existiert, mit \(g \circ f = Id_X\)}
\item{f ist genau dann surjektiv, wenn \(g : Y \to X\) existiert, mit \(f \circ g = Id_Y\)}
\item{f ist genau dann bijektiv, falls \(g : Y \to X\) existiert, so dass sowohl \(g \circ f = Id_X\) und \(f \circ g = Id_Y\). Es gilt dann \(g = f^{-1}\)}
\end{enumerate}
\textbf{Beweis}
\begin{enumerate}
\item{Sei f injektiv. Dann existiert zu jedem \(y \in f(X)\) genau ein \(x \in X\) mit \(f(x) = y\). Wir setzen \(g(y) = x\) f\"ur ebensolche \(y = f(x)\). Nun w\"ahlen wir \(x_0\in X\) beliebig und setzen \(g(y') = x_0\) f\"ur alle \(y' \in \setminus f(X)\). Dieses g erf\"ullt die Bedingung.\\
Sei nun \(g : Y \to X\) mit \(g \circ f = Id_X\)}. Seien \(x, x' \in X\) mit \(f(x) = f(x')\). Es gilt \(x = Id_X(x) = (g \circ f)(x) = g(f(x)) = g(f(x')) = (g \circ f)(x') = Id_X(x') = x'\). Also ist f injektiv.
\item{Sei f surjektiv. Zu jedem \(y \in Y\) w\"ahlen wir ein \(x \in X\) mit \(f(x) = y\) und setzen \(g(y) = x\). Damit gilt \(f \circ g = Id_Y\).\\
Ungekehrt, sei \(g : Y \to X\), so dass \(f \circ g = Id_Y\). Sei \(y \in Y\), dann gilt \(y = f(g(y))\). Sei \(x' = g(y)\). Damit ist \(y = f(x'), x' \in X\) und \(y \in f(X)\). Damit ist f surjektiv.}
\item{Sei f bijektiv. Die nun definierte Abbildung \(f^{-1} : Y \to X\) erf\"ullt die Voraussetzung an g.\\
Falls aber g existiert, mit \(g \circ f = Id_x\) und \(f \circ g = Id_Y\). Damit erf\"ullt g die Voraussetzungen von 1. und 2. und f ist sowohl injektiv als auch surjektiv. Es gilt dann auch \(g = f^{-1}\).}
\end{enumerate}

\subsubsection{Definition 1.14 Menge aller Abbildungen}
Seien X, Y Mengen. Mit \(Abb(X, Y)\) bezeichnen wir die Menge aller Abbildungen von X nach Y.\\
\textbf{Bemerkung}\\
\(\{ f \in Abb(X, Y) : f \text{ surjektiv}\}\) ist nun ebenfalls definiert.

\subsubsection{Definition 1.15 M\"achtigkeit von Mnegen}
Es seien X, Y Mengen. Wir sagen X ist gleichm\"achtig wie Y, falls eine bijektive Abbildung von X nach Y existiert.\\
\textbf{Bemerkung}\\
F\"ur endliche  Mengen M gilt \(\#M = m\) genau dann, wenn M gleichm\"achtig wie \(\{1, 2, \dots, m\}\) ist.

\subsubsection{Definition 1.16 Potenzmenge}
Sei M eine Menge. Die Menge aller Teilmengen von M hei\ss{}t Potenzmenge von M, kurz \(2^M\).\\
\textbf{Bemerkung}\\
F\"ur eine (beliebige nicht notwendigerweise bijektive) Abbildung \(f : X \to Y\) ist \(f^{-1}\) eine Abbildung von \(2^Y\) nach \(2^X\).

\subsubsection{Satz 1.17 M\"achtigkeit von \(2^M\)}
Sei M eine endliche Menge mit \(\#M = m\), \(m \in \N \cup \{0\}\). Dann gilt \(\#2^M = 2^m\).\\
\textbf{Beweis}\\
F\"ur m = 0 gilt \(M = \emptyset\) und die Aussage ist klar. (denn \(2^\emptyset = \{\emptyset\}\), und diese Menge besitzt ein Element).\\
Rest des Beweises mittel Induktion.\\
Wir nennen \(K \subset \N\) die Menge der nat\"urlichen Zahlen m, f\"ur welche die Aussage gilt, und zeigen:\\
1.) \(1 \in K\)\\
2.) falls \(m \in K\) so ist auch \(m + 1 \in K\).\\
Damit folgt (nach dem Induktionsprinzip), dass  K = \mN\ und der Satz ist gezeigt.\\
Zu 1.) Die einelementige Menge M schreiben wir als \(\{x\}\), die Teilmengen sind \(\emptyset, \{x\}\). Somit ist \(2^M = \{\emptyset, \{x\}\}\) mit \(\#2^M = 2 = 2^1\).\\
Zu 2.) Es sei also \(\#M = m + 1\)  und \(M_m\) eine Menge mit \(\# M_m = m\). Wir d\"urfen annehmen, dass gilt \(\#2^{M_m} = 2^m\). Wir schreiben M als \(M_m \cup \{x\}, x \not\in M_m\). Wir scheiben\\
\(2^M = \{\)Menge aller Teilmengen von M, welche x nicht enthalten\(\} \cup \{\)Menge aller Teilmengen von M, welche x enthalten\(\} = A \cup B\) und es gilt \(\#2^M = \#A + \#B\).\\
\(\#A = \#2^{M_m} = m\), da \(A = 2^{M_m}\).\\
Jede Menge in B ist aber eine Menge in \(2^{M_m}\) vereinigt mit \(\{x\}\) und \(\#B = 2^m\). Somit gilt \(\#2^M = 2^m + 2^m = 2^{m + 1}\).\\
Damit gilt die Aussage f\"ur \(m + 1\).\\
\\
Wir kennen bereits das direkte (bzw. kartesische) Produkt zweier Mengen \(X \times Y = \{(x, y) : x\in X, y \in Y\}\).\\

\subsubsection{Definition 1.18 Graph einer Funktion}
Es sei \(f : X \to Y\) eine Abbildung. Die Menge \(\Gamma_f = \{(x, f(x)) \in X \times Y\}\) nennen wir Graph von f.\\
\\
Noch n\"utzlicher ist das direkte Produkt um eine sogenannte Relation zu definieren.\\
\textbf{Beispiele}\\
\(x \underset{\text{(Steht in Relation zu)}}{\sim} y \Leftrightarrow x \le y\)\\

\subsubsection{Definition 1.19 Relationen}
Eine Relation R auf einer Menge X ist eine Teilmenge von \(X \times X\).\\
 Wir sagen f\"ur \(x, y \in X\), dass \(x \sim y\) genau dann, wenn \((x, y) \in R\).\\
\\
F\"ur das Beispiel gilt \(R = \{(x, y) \in X \times X : x \le y\}\)\\

\subsubsection{Definition 1.20 \"Aquivanlenzrelation}
Eine Relation \(\sim\) auf X hei\ss{}t \"Aquivalenzrelation, falls gilt:
\begin{enumerate}
\item{\(x \sim x\) (Reflexivit\"at)}
\item{\(x \sim y \Rightarrow y \sim x\) (Symmetrie)}
\item{\(x \sim y \land y \sim z \Rightarrow x \sim z\) (Transitivit\"at)}
\end{enumerate}
f\"ur alle \(x, y, z \in X\)

\textbf{Beispiele:}\\
''\(=\)'' auf Zahlensystemen.\\
\(X = 2^N \). F\"ur \( x, y \in X\) gelte \(x \sim y\) falls endliche Teilmengen A, B von x und y mit \(x \setminus A = y \setminus B \)

\subsubsection{Definition 1.21 \"Aquivalenzklassen}
Sei X eine Menge mit \"Aquivalenzrelation ''\(\sim\)''. Eine Menge \(A \subset X\) hei\ss{}t \"Aquivalenzklasse bez\"uglich ''\(\sim\)'', falls gilt:
\begin{enumerate}
\item{\(A \neq \emptyset\)}
\item{falls \(x, y \in A \Rightarrow x \sim y\)}
\item{\(x \in A, y \in X, x \sim y \Rightarrow y \in A\)}
\end{enumerate}

\subsubsection{Proposition 1.22 Partitionierung in \"Aquivalenzklassen}
Sei X eine Menge mit \"Aquivalenzrelation ''\(\sim\)''. Dann geh\"ort jedes \(a \in X\) zu genau einer \"Aquivalenzklasse A bez\"uglich ''\(\sim\)''.\\
F\"ur zwei \"Aquivalenzlkassen A, A' gilt entweder \(A = A'\) oder \(A\cap A' = \emptyset\).\\
\\
\textbf{Beweis}\\
F\"ur \(a \in X\) definieren wir die Menge \(A = \{x \in X : a \sim x\}\).\\
Nachdem \(a \sim a\) gilt \(a \in A\), somit ist \(A \neq \emptyset\). Sind nun \(x, y \in A\), so gilt \(a \sim x \land a \sim y\). Damit folgt \(x \sim a\) und \(a \sim y\) und somit \(x \sim y\).\\
F\"ur \(x \in A, y \in X\) mit \(x \sim y\). gilt \(a \sim x, x \sim y\) also \(a \sim y\) und somit \(y \in A\).\\
Somit ist A eien \"Aquivalenzklasse und a ist in \textit{mindestens} einer \"Aquivalenzklasse enthalten.\\
Es ist noch zu zeigen, dass zwei \"Aquivalenzklassen entweder gleich oder disjunkt sind.\\
Seien also \(A, A'\) \"Aquivalenzklassen mit \(A \cap A' \neq \emptyset\). Also existiert \(b \in A\cap A'\). Falls nun \(x \in A\), so gilt \(x \sim b\). Nachdem b auch in A' ist, folgt aber \(x \in A'\). Damit folgt \(a \subset A'\). Die Umkehrung, also \(A' \subset A\) folgt ebenso.

\subsubsection{Definition 1.23 Quotientenmenge}
Es sei X eine Menge mit \"Aquivalenzrelation ''\(\sim\)''. Die Menge der \"Aquivalenzklassen in X bezeichnen wir  als Quotientenmenge und schreiben
 f\"ur diese Menge \(X/{\sim}\)\\
 \\
 \textbf{Bemerkung}\\
 Wir k\"onnen eine Abbildung definieren, welche jedem \(a \in X\) dessen \"Aquivalenzklasse zuordnet:\\
 \(X \to x/{\sim}, a \mapsto A_a\) (nach Prop 1.22 eindeutig zugeorndete \"Aquivalenzklasse).\\
 Ein solches a hei\ss{}t dann Repr\"asentant der \"Aquivalenzklasse \(A_a\).\\
 \\
 \textbf{Beispiel}\\
 Sei X = \mN. Wir schreiben \(X \sim y\), falls sowohl x als auch y gerade bzw. ungerade Zahlen sind.\\
 Sei \(a \in X\). Die zugh\"orige \"Aquivalenzklasse ist gegeben durch:\\
 Die Menge aller geraden Zahlen, falls a gerade ist.\\
 Die Menge aller ungeraden Zahlen, falls a ungerade ist.

\subsection{Gruppen}

\subsubsection{Definition 1.24 Verkn\"upfungen}
Es sei G eine Menge. Eine Verkn\"upfung \(*\) auf G ist eine Abbildung:\\
\(* : G \times G \to G\).\\
\((a, b) \mapsto *(a, b)\)\\
\\
\textbf{Bemerkung} Oft schreiben wir einfach \(a * b\) f\"ur \(*(a, b)\).\\
\textbf{Beispiele}\\
\(G = \N, *(a, b) = a \cdot b\)\\
\(G = \N, *(a, b) = a + b\)\\
X Menge, \(G = Abb(X, X), *(f, g) = f \circ g\)

\subsubsection{Definition 1.25 Gruppen}
Eine Menge G mit Verkn\"upfung \(*\) hei\ss{}t Gruppe, falls gilt:
\begin{enumerate}
\item{\((a * b) * c = a * (b * c)\) (Assoziativgesetzt)}
\item{Es existiert ein Element \(e \in G\), so dass gilt:
\begin{enumerate}
\item{\(a * e = a\) f\"ur alle \(a \in G\)}
\item{F\"ur alle \(a \in G\) existiert \(a' \in G\) mit \(a' * a = e\)}
\end{enumerate}
}
\end{enumerate}
Die Gruppe hei\ss{}t abelsch, falls zus\"atzlich gilt \(a * b = b * a\) f\"ur alle \(a, b \in G\).\\
e aus 2. a.) hei\ss{}t neutrales Element.\\
a' ais 2. b.) hei\ss{}t inverses Element.\\
\\
\textbf{Bemerkung}\\
Wir schreiben oft einfach \(a \cdot b\) bzw. \(ab\) f\"ur \(a * b\).\\
\textbf{Beispiele}
\begin{enumerate}
\item{\(G = \Z, *(a, b) = a + b\) (\(e = 0, a' = -a)\)}
\item{\(G = \Q \setminus\{0\}, *(a, b) = a \cdot b\) (\(e = 1, a' = \frac{1}{a}\))}
\item{\(G = \{f \in Abb(X, X), f\) bijektiv\(\}, *(f, g) = f \circ g\) (\(e = Id_X, f^{-1}\) als inverses)}
\end{enumerate}
Achtung: 1 und 2 sind abelsch, 3 nicht notwendigerweise.

\subsubsection{Propostition 1.26 Eindeutigkeit neutrales und inverses}
Es sei G eine Gruppe. Dann gilt
\begin{enumerate}
\item{Das neutrale Element ist eindeutig bestimmt, und es gilt auch \(a * e = a\)}
\item{Das inverse Element a' ist zu jedem \(a \in G\) eindeutig bestimt und es gilt auch \(a * a' = e\)}
\end{enumerate}
\textbf{Beweis}\\
Wir betrachten ein \(e \in G\) und ein \(a \in G\), wobei e ein neutrales Element ist. Es sei \(a'\) ein Inverses zu a. Es folgt:\\
\(a a' = e (a a') = (a'' a') (a a') = a'' (a' (a a')) = a'' ((a' a) a') = a'' (e a') = a'' a' = e\)\\
Somit gilt \(a e = a (a' a) = (a a') a = a\).\\
Sei \(\hat{e}\) ein anderes neutrales Element. Dann gilt \(e \hat{e} = e\) und \(e \hat{e} = \hat{e}\). Damit folgt \(e = \hat{e}\).\\
Sei nun \(\hat{a}'\) ein weiteres inverses Element, dann folgt:\\
\(\hat{a}' = \hat{a}' e = \hat{a}' (aa') = (\hat{a}'a)a' = ea' = a'\)\\
\\
\textbf{Bemerkungen}
\begin{enumerate}
\item{
Wir schreiben \(a^{-1}\) f\"ur das (nun) eindeutig bestimmte inverse Element zu a.\\
Es gilt also \(a^{-1}a = aa^{-1} = e\) sowie \((a^{-1})^{-1} = a\) und \((ab)^{-1} = b^{-1}a^{-1}\)\\
(denn \(b^{-1}a^{-1})(ab) = b^{-1}((a^{-1}a)b) = b^{-1}(eb) = b^{-1}b = e\)
}
\item{
Es folgen auch die K\"urzungsregeln:\\
\(a \hat{x} = ax \Rightarrow x = \hat{x}\)\\
und \(\hat{y}a = ya \Rightarrow y = \hat{y}\)
}
\end{enumerate}

\subsubsection{Definition 1.27 Rechts- und Linkstranslation}
F\"ur \(a \in G\), G eine Gruppe, schreiben wir\\
\(\tau_a : G \to G, x \mapsto x a\) (Rechtstranslation)\\
\(_{a}\tau : G \to G, x \mapsto a x\) (Linkstanslation)

\subsubsection{Lemma 1.28}
\begin{enumerate}
\item{Falls G eine Gruppe ist, so sind \(\tau_a\) und \(_{a}\tau\) bijektiv.}
\item{Sei G eine Menge mit assoziativer Verkn\"upfung. Dann folgt Def 1.25 2. aus surjektivit\"at von \(t_a\) und \(_{a}\tau\)}
\end{enumerate}
\textbf{Beweis}\\
\begin{enumerate}
\item{
Bijektivit\"at folgt aus  \((\tau_a)^{-1}\) gegeben durch \((\tau_a)^{-1}(x) = x a^{-1}\), denn \((tau_a)^{-1}(\tau_a(y)) = \tau_a(y)a^{-1} = (y a) a^{-1} = y\) f\"ur jedes \(y \in G\).
}
\item{
 Seien also \(\tau_a\) und \(_{a}\tau\) surjektiv. Dann existiert f\"ur jedes \(b \in G\) eine L\"osung f\"ur:\\
 \(x a = b\) sowie \(a y = b\).\\
 Damit existiert aber zu \(a \in G\) ein  e mit \(ea = a\). F\"ur beliebiges \(b \in G\) folgt dann \(e b = e (a y) = (e a) y = ay = b\)\\
 Durch L\"osen von \(x a = e\) bekommen wir analog das Inverse Element zu a.
}
\end{enumerate}
\textbf{Bemerkungen}
\begin{enumerate}
\item{Falls die Gefahr der Verwechslung besteht, scheiben wir gerne \((G, *)\) Ff\"ur eine Gruppe G mit Verkn\"upfung \(*\).\\
Beispielsweise \((\Q, +)\) f\"ur \mQ mit Addition, oder \((\Q \setminus {0}, \cdot)\) f\"ur \(\Q \setminus {0}\) mit Multiplikation.\\}
\item{Bei der Verkn\"upfung + gehen wir immer von kommutativit\"at aus.}
\item{Endliche Gruppen kann man mit einer (Gruppen-) Tafel darstellen\\
\begin{tabular}{c || c | c | c}
* & e & \(\cdots\) & \(a_i\)\\
\hline
\hline
e & e &        & \(a_i\)\\
\hline
\(\vdots\) & & \\
\hline
\(a_j\) & \(a_j\)& & \(a_i * a_j\)
\end{tabular}
}
\item{
Es gibt nur eine zweielementige Gruppe:\\
\begin{tabular}{c || c | c}
* & e & a\\
\hline
\hline
e & e & a\\
\hline
a & a & e
\end{tabular}
}
\end{enumerate}

\subsubsection{Definition 1.29 Untergruppen}
Es sei \((G, \cdot)\) eine Gruppe, \(G' \subset G\). \(G'\) hei\ss{}t Untergruppe von G, falls f\"ur \(a, b \in G'\) auch gilt \(ab \in G'\) und \(a^{-1} \in G'\).

\subsubsection{Definition 1.30 Homo- und Isomorphismen auf Gruppen}
Seien \((G, \cdot), (H, *)\) Gruppen, und \(\varphi : G \to H\) eine Abbildung.
\begin{enumerate}
\item{
Die Abbildung \(\varphi\) hei\ss{}t Homomorphismus, falls gilt, dass\\
\(\varphi(a \cdot b) = \varphi(a) * \varphi(b)\) f\"ur alle \(a, b \in G\)
}
\item{
\(\varphi\) hei\ss{}t Isomorphismum, falls \(\varphi\) zus\"atzlich bijektiv ist.
}
\end{enumerate}

\subsubsection{Proposition 1.31 Untergruppen sind Gruppen}
Es sei \((G, \cdot)\) eine Gruppe, \(G'\) eine Untergruppe von G. Dann ist \((G', \cdot)\) selbst eine Gruppe.\\
\\
\textbf{Beweis}\\
Assoziativit\"at folgt sofort. Es ex \(a^{-1}\) in \(G'\), somit auch \(e = aa^{-1} \in G'\).

\subsubsection{Proposition 1.32 Eigenschaften von Homomorphismen}
Sei \(\varphi : G \to H\) ein Homomorphismus von Gruppen \((G, \cdot), (H, *)\). Dann gilt
\begin{enumerate}
\item{\(\varphi(e) = \hat{e}\) mit neutralen Elementen \(e \in G, \hat{e} \in H\)}
\item{\(\varphi(a^{-1}) = (\varphi(a))^{-1}\) f\"ur alle \(a \in G\)}
\item{F\"ur einen Isomorphismus \(\varphi\) ist auch \(\varphi^{-1}\) ein Homomorphismus}
\end{enumerate}

\textbf{Beweis}
\begin{enumerate}
\item{
\(\hat{e} * \varphi(e) = \varphi(e) = \varphi(e \cdot e) = \varphi(e) * \varphi(e)\)\\
Nach der K\"urzungsregel folgt \(\hat{e} = \varphi(e)\)
}
\item{
Aus 1. gilt \
\(\hat{e} = \varphi(e) = \varphi(a^{-1} a) = \varphi(a^{-1}) * \varphi(a)\) also ist \(\varphi(a^{-1}) = (\varphi(a))^{-1}\)
}
\item{
Wir betrachten \(c, d \in H\) mit \(c = \varphi(a), d = \varphi(b)\). Dann gilt\\
\(\varphi(a b) = \varphi(a) * \varphi(b) = c * d\)\\
also \(\varphi^{-1}(c * d) = \varphi^{-1}(\varphi(a b)) = ab = \varphi^{-1}(c) \varphi^{-1}(d)\)
}
\end{enumerate}

\textbf{Beispiele}
\begin{enumerate}
\item{
\(G = (\R, +), H = (\{x \in \R : x > 0\}, \cdot)\)\\
\(exp : \R \to \R^+_*, x \mapsto e^x\)\\
ist ein Isomorphismus, denn \(e^{x + y} = e^x e^y\).
}
\item{
Wir betrachten \((\Z, +)\). Sei \(m \in \Z\). Dann ist \(\varphi_m : \Z \to \Z, a \mapsto ma\) ein Homomorphismus, denn \(m(a + b) = ma + mb\).\\
Das Bild \(\phi_m(\Z) = m\Z = \{m a : a \in \Z\} \subset \Z\)  ist eine Untergruppe von \((\Z, +)\), denn \(ma + mb = m(a + b) \in m\Z\) und \(-(ma) = m(-a) \in m\Z\).\\
Dazu betrachten wir die Menge \(r + m\Z\) (f\"ur \(r \in \{0, 1, \dots, m-1\}\)) mit \(r + m\Z = \{r+ma : a \in \Z\}\). Dann gilt \(\Z = (0 +m\Z) \cup (1 + m\Z) \cup \dots \cup(m-1 \cup m\Z)\) und die Vereinigung ist disjunkt.\\
F\"ur \(a \in \Z\) gilt \(\frac{a}{m} = k + \frac{r}{m}\) f\"ur \(k \in \Z, r \in \{0, \dots, m-1\}\) (Division mit Rest).\\
Dann gilt \( a \in r + m\Z\). (denn \(a = km + r\)).\\
Wir beseichnen die Mengen \(r + m\Z\) auch als sogenannte ''\textit{Restklassen modulo m}''.\\
Falls \(a, a'\) in derselben Klasse \(r +m\Z\) sind, gilt \(\frac{a-a'}{r} \in \Z\), und wir schreiben \(a \equiv a' \mod m\) (ist kongruent zu).\\
Zu \(a \in \Z\) schreiben wir \(\bar{a} = a + m\Z\), die zu a geh\"orige Restklasse und wir definieren eine Addition \(\bar{a} + \bar{b} = \overline{a + b}\).\\
Wir m\"ussen sicherstellen, dass die Definition nicht von der Auswahl des Repr\"asentanten abh\"angt, das ist aber leicht zu sehen.\\
\(\bar{a} = \bar{a'}, \bar{b} = \bar{b'}\), dann folgt auch schon, dass gilt \(\overline{a + b} = \overline{a' + b'}\).
}
\end{enumerate}

\subsubsection*{Satz}
F\"ur \(m \in \N\)  sei \(Z/m\Z = \{\bar{0}, \dots, \overline{m-1}\}\).\\
Dann gilt, dass \(\Z/m\Z, +\) (+ def. wie oben) eine abelsche Gruppe ist.\\
Die Abbildung \(Z \to Z/m\Z, a \mapsto \bar{a} = a + m\Z\) ist ein surjektiver Homomorphismus.\\
Beweis: \"Ubung.\\
Wir nennen diese Gruppen die zyklischen Gruppen der Ordnung m.

\subsection{Ringe und K\"orper}

\subsubsection{Def 1.33 Ringe}
Es sei R eine Menge, \(+ : R \times R \to R\) und \(\cdot : R \times R \to R\) Verkn\"upfungen. \((R, +, \cdot)\) hei\ss{}t Ring, falls:
\begin{enumerate}
\item{\((R, +)\) ist eine abelsche Gruppe}
\item{Die Multiplikation  \(\cdot\) assoziativ ist.}
\item{Das Distributivgesetz gilt:\\
\(a \cdot (b + c) = ab + ac\)\\
\((b + c) \cdot a = ba + ca\)
}
\end{enumerate}
Ein Ring hei\ss{}t kommutativ, falls gilt \(a \cdot b = b \cdot a\) f\"ur alle \(a, b \in R\).\\
Falls ein Element \(1 \in R\) existiert mit \(1 \cdot a = a \cdot 1 = a\) f\"ur alle \(a \in R\), dann nennen wir dieses Element Einselement.\\
Das neutrale Element der Addition + hei\ss{}t Nullelement (oder 0).

\subsubsection{Proposition 1.34 Absorption durch Nullelement}
Es gilt \(0 \cdot a = a \cdot 0 = 0\).\\
\textbf{Beweis}\\
Wir erinnern uns an die K\"urzungsregel: \(\al + \xi = \beta + \xi \Rightarrow \al = \beta\).\\
Wir schreiben \(0 + 0a = 0a = (0 + 0) a = 0a + 0a \Rightarrow 0 = 0a\)\\
Ebenso folgt \(0 = a0\).\\
\\
\textbf{Beispiele}
\begin{enumerate}
\item{\((\Z, + \cdot)\), \((\Q, +, \cdot), (\R, + \cdot)\).}
\item{\(\Z.m\Z\) mit + wie bisher und \(\bar{a} \cdot \bar{b} = \overline{ab}\) (Nach \"Uberpr\"ufung der Unabh\"angigkeit von der Wahl des Repr\"asentanten)}
\end{enumerate}
\textbf{Beispiel}\\
Die 2x2-Matrizen \(A = \begin{pmatrix}a & b\\c & d\end{pmatrix}\) bilden einen Ring mit\\
\(\begin{pmatrix}a & b\\c & d\end{pmatrix} + \begin{pmatrix}e & f\\g & h\end{pmatrix} = \begin{pmatrix}a + e & b + f\\c + g & d + h\end{pmatrix}\)\\
\(\begin{pmatrix}a & b\\c & d\end{pmatrix} \cdot \begin{pmatrix}e & f\\g & h\end{pmatrix} = \begin{pmatrix}ac+bg & af+bh\\ce+dg & cf+dh\end{pmatrix}\)\\
Die gew\"unschten Eigenschaften folgen sofort. Es gilt aber:\\
\(\begin{pmatrix}1 & 1\\0 & 1\end{pmatrix} \begin{pmatrix}1 & 0\\1 &1\end{pmatrix} \neq \begin{pmatrix}1 & 0\\1 & 1\end{pmatrix} \begin{pmatrix}1 & 1\\0 & 1\end{pmatrix}\)

\subsubsection{Definition 1.35 Unterring und Ringhomomorphismus}
\begin{enumerate}
\item{
Es sei \((R, +, \cdot)\) ein Ring, \(R' \subset R\). \((R', +, \cdot)\) hei\ss{}t Unterring, falls \((R', +)\) eine Untergruppe von \((R, +)\) ist und gilt \(a, b \in R' \Rightarrow ab \in R'\)
}
\item{
Es seien \((R, +, \cdot)\), \((S, \hat{+}, \hat{\cdot})\) Ringe, \(\varphi : R \to S\) eine Abbildung. \(\varphi\) hei\ss{}t Ringhomomorphismus, falls gilt \(\varphi(a + b) = \varphi(a) \hat{+} \varphi(b)\) und \(\varphi(ab) = \varphi(a) \hat{\cdot} \varphi(b)\) f\"ur alle \(a, b \in R\).
}
\end{enumerate}

\subsubsection{Definition 1.36 K\"orper}
Es sei K eine Menge, \(+ : K \times K \to K, \cdot : K \times K \to K\) Verkn\"upfungen. \((k, +, \cdot)\) hei\ss{}t K\"orper, falls gilt:
\begin{enumerate}
\item{\((K, +)\) ist eine abelsche Gruppe}
\item{\(K^*\) sei gegeben durch \(K \setminus \{0\}\). Dann gilt \((K^*, \cdot)\) ist eine abelsche Gruppe.}
\item{F\"ur \(a, b, c \in K\) gilt  \(a (b+c) = ab + bc\) und \((b+c) a = ba + ca\)}
\end{enumerate}
\textbf{Bemerkung}\\
Das neutrale Element der Multiplikation bezeichnen wir mit Eins \(( = 1)\), das Inverse zu a bez\"uglich der Multiplikation mit \(a^{-1}\) oder \(\frac{1}{a}\), bez\"uglich der Addition mit \(-a\).

\subsubsection{Proposition 1.37 Rechenregeln f\"ur K\"orper}
Sei \((K, +, \cdot)\) ein K\"orper. Dann gilt
\begin{enumerate}
\item{\(1 \neq 0\)}
\item{\(0a = a0 = 0\)}
\item{\(ab = 0 \Rightarrow a = 0 \lor b = 0\)}
\item{\(a(-b) -(ab)\) und \((-a)(-b) = ab\)}
\item{\(x a = \hat{x}a\) und \(a \neq 0 \Rightarrow x = \hat{x}\)}
\end{enumerate}
\textbf{Beweis}
\begin{enumerate}
\item{Folgt sofort, denn \((K^*, \cdot)\) ist eine Gruppe.}
\item{Folgt analog zu Ringen.}
\item{Folgt aus Gruppeneigenschaft von \((K^*, \cdot)\), da \((K^*, \cdot)\) unter der Multiplikation abgeschlossen ist, und somit a oder b nicht in \(K^*\) sein kann (also 0 ist)}
\item{Wir rechnen \\
\(ab + a(-b) = a (b - b) = a 0 = 0\)\\
und\\
\((-a)(-b) = -((-a)b) = -(-(ab)) = ab\)}
\item{Die Regel gilt f\"ur \(x, \hat{x}\) beide in \(K^*\). Ist aber \(\hat{x} = 0\), so gilt \(\hat{x}a = 0\) nach 2. und mit 3. folgt die Aussage.}
\end{enumerate}
\textbf{Beispiele}
\begin{enumerate}
\item{\((\Q, +, \cdot), (\R, +, \cdot)\).}
\item{Die komplexen Zahlen \(\C\), wie folgt definiert. F\"ur \((a, b), (c, d) \in \R \times \R\) sei\\
\((a, b)+(c, d) = (a + c, b + d)\) und\\
\((a,b)\cdot (c,d) = (ac-bd, ad + bc)\).\\
Mit \((0, 0)\) als Nullelement und \((1, 0)\) als Einselement.\\
Das additive Inverse zu \((a, b)\) ist dann \((-a, -b)\), das Multiplikative Inverse ist \((\frac{a}{a^2+b^2}, -\frac{b}{a^2 + b^2})\)\\
Wir bezeichnen den so konstruierten K\"orper mit \(\C\).\\
Die Abbildung \(\R \to \C, a \mapsto (a, 0)\) ist injektiv. Wir sehen, dass zwischen \(\R \times \{0\}\) und \(\{(a, b) \in \C : b = 0\}\) nicht unterschieden werden muss, denn \\
\(
(a, 0) (b, 0) = (ab, 0)\\
(a, 0) + (b, 0) = (a+b, 0)
\)\\
Wir schreiben \(i = (0, 1) \in \C\) und \((a, b) = (a, 0) + (0, b) = a + ib\).\\
Es gilt \(i^2 = ii = -1\). Weiterhin schreiben wir  f\"ur \(z = (a, b) \in \C, \bar{z} = (a, -b)\). (bzw. \(z = a + ib, \bar{z} = a - ib\)). (Komplex konjugiertes)\\
F\"ur komplexe Zahlen \(\lb, \mu\) gilt dann \(\overline{\lb + \mu} = \bar{\lb} + \bar{\mu}\) sowie \(\overline{\lb \mu} = \bar{\lb}\bar{\mu}\) und \(\lb \in R \Leftrightarrow lb = \bar{\lb}\)\\
F\"ur \(\lb = a+ib \in \C\) sehen wir \(\lb \bar{\lb} = (a + bi) (a - bi) = a^2 + b^2 \in \R\) und wir definieren den Absolutbetrag \(|\lb| = \sqrt{\lb \bar{\lb}}\).\\
Damit gilt, dass \(d(\lb, \mu) = |\lb - \mu|\) ein Metrik im Sinne von Kap 0 darstellt. (denn \(d(\mu, \lb) = d(\lb, \mu),\\
d(\mu, \lb) = 0 \Leftrightarrow \lb = \mu,\\
d(\mu, \lb) + d(\lb, \kappa) \ge d(\mu, \kappa)\))\\
(Das ist die selbe Metrik, die bereits im \(\R^2\) eingef\"uhrt wurde.)\\
\(d(x, y) = \sqrt{(x - y, x - y)} = \sqrt{(x_1-y_1)^2 + (x_2 - y_2)^2}\) mit (\((\xi, \eta) = \xi_1 \eta_1 + \xi_2 \eta_2\)).\\
Neu ist die Identit\"at \(|\lb \cdot \mu| = |\lb| |\mu|\)\\
Wir betrachten noch eine geometrische Anschauung der komplexen Zahlen. Es sei \(\lb \in \C\) mit \(|\lb| = 1\). Dann gilt, dass \(\lb^{-1} = |\frac{1}{\lb}| = 1\)  (folgt aus der Formel f\"ur das Inverse bezgl. Multiplikation in \mC).\\
In der Analysis lernen wir, dass gilt:\\
- es existiert ein eindeutiges \(\al \in [0, 2\pi)\), so dass \(\lb = cos(\al) + i \sin(\al) = e^{i\al}\) f\"ur \(\lb \in \C, |\lb| = 1\).\\
Wir bezeichnen \(\al\) als Argument von \(\lb\), also \(\al = \arg \lb\).\\
Sei nun \(\lb \in \C \setminus 0\) beliebig (d.h. ohne die Einsch\"ankung, dass \(|\lb| = 1\)). Dann schreiben wir \(\arg \lb = \arg \frac{\lb}{|\lb|}\), denn \(|\frac{\lb}{|\lb|}|\).\\
Damit gilt \(\lb = |\lb| e^{i \arg \lb}\) (f\"ur jedes \(\lb \in \C\)).\\
In der komplexen Ebene \(\C = \R^2\) gilt dann:\\
\begin{tikzpicture}
\draw[->] (-1, 0) to (6, 0);
\draw[->] (0, -1) to (0, 6);
\node() at (3, 5) {(a + ib)};
\node() at (3, -0.5) {a};
\node() at (-0.5, 5) {b};
\draw[color=red] (0, 0) to (3, 5);
\node[color=red] at (1.5, 3) {d};
\node() at (1.5, 1) {\(\alpha\)};
\node() at (6, -0.5) {\mR};
\node() at (-0.5, 6) {i\mR};
\draw[] (3, 0) to [bend right=20] (1.5, 2.5);
\end{tikzpicture}\\
mit \(d = |\lb|, \al = \arg \lb\).\\
Wir sehen nun, dass gilt \(\lb \mu = |\lb| e^{i \arg \lb} \cdot  |\mu| e^{i \arg \mu} = |\lb||\mu| e^{e \arg\lb} e^{i \arg \mu} = |\lb||\mu| e^{i (\arg\lb + \arg \mu)}\).\\
D.h. Betr\"age werden multipliziert, Argumente addiert bei der Multiplikation in \mC.
}
\end{enumerate}

\subsubsection{Definition 1.38 Nullteilerfreiheit von Ringen}
Ein Ring \((R, +, \cdot)\) hei\ss{}t Nullteilerfrei, falls f\"ur \(a, b \in R\) gilt \(ab = 0 \Rightarrow a = 0 \lor b = 0\).\\
\\
\textbf{Bemerkung}\\
Wir sehen, dass jeder K\"orper bereits ein nullteilerfreier Ring ist.\\
\textbf{Beispiele}\\
Auf \(\Z/m\Z\) ist bereits eine Addition definiert, mit der \(\Z/m\Z\) eine Gruppe wird. Mit der Multiplikation\\
\(\bar{a} \cdot \bar{b} = \overline{ab}\)\\
f\"ur \(\bar{a}, \bar{b} \in \Z / m\Z\) und Repr\"asentanten a und b wird \(\Z/m\Z\) zu einem Ring\\
Wie f\"ur die Addition zeigen wir unabh\"angigkeit von der Wahl der Repr\"asentanten.\\
(Assoziativit\"at und Distributivgesetz sind leicht Nachzurechnen.) Der Ring ist kommutativ.

\subsubsection{Satz 1.39 Nullteilerfreiheit des Restklassenrings}
Der Restklassenring \((\Z/m\Z, +, \cdot)\) ist genau dann nullteilerfrei, wenn m eine Primzahl ist.\\
\\
\textbf{Beweis}\\
Falls m nicht prim ist, gilt \(m = k \cdot l\) mit \(1 < k, l < m\). Damit gilt \(\bar{k} \neq \bar{0}, \bar{l} \neq \bar{0}\), aber \(\bar{k} \bar{l}= \overline{kl} = \bar{m} = \bar{0}\).\\Umgekehrt: Sei m prim und \(\bar{k} \bar{l} = 0\). Dann gilt \(k \cdot l= r \cdot m\), f\"ur ein \(r \in \Z\). Damit gilt aber, dass mindestens einer der Faktoren \(k, l\) einen Faktor m enth\"alt. Also ist \(\bar{k} = 0\) oder \(\bar{l} = 0\).

\subsubsection{Satz 1.40 Ringe dei K\"orper sind}
Ein nullteilerfreier, kommutativer Ring K mit endlich vielen Elementen und Eins ist ein K\"orper.\\
\\
\textbf{Beweis}\\
Nach Lemma 1.28 reicht es zu zeigen, dass die Abbildung \({}_{a}\tau : K^* \to K^* : {}_{a}\tau(x) = ax\) f\"ur jedes \(a \in K^*\) surjektiv ist. \(K^*\) ist eine endliche Menge, also folgt surjektivit\"at aus injektivit\"at. Sei also \({}_a\tau(x) = {}_a\tau(y)\), f\"ur x, y aus \(K^*\). Es folgt \(ax = ay\), also \(a (x - y) = 0\). Damit gilt aber  (wegen Nullteilerfreiheit und \(a \in K^*\), also \(a \neq 0\)), dass \(x - y = 0\), also \(x = y\).

\subsubsection{Definition 1.41 Charakteristik eines Ringes}
Es sei R ein Ring mit Einselement 1. Die Charakteristik  von R ist gegeben durch\\
\(\chi(R) = \begin{cases}
0 & \text{, falls } n \cdot 1 \neq 0 \forall n \neq 0\\
min(n \in \N \setminus \{0\}) : n \cdot 1 = 0
\end{cases}\)\\
Statt \(\chi(R)\) wird auch \(char(R)\) verwendet.\\
\\
\textbf{Achtung:} Wir haben benutzt, dass \(n \cdot a = a + a + \dots + a\) (n-mal) mit \(a \in R, n \in \N\)

\subsubsection{Lemma 1.42 Charakteristik von K\"orpern}
Ist K ein K\"orper, so gilt \(\chi(K)\) ist entweder  Null, oder eine Primzahl.\\
\\
\textbf{Beweis}\\
Angenommen, \(\chi(K) = m = k \cdot l \neq 0\) mit  \(1 < k, l < m\) (also m keine Primzahl). Es folgt  \(0 = m \cdot 1 = (k \cdot l) 1 = (k \cdot 1)(l \cdot 1)\). Wegen Nullteilerfreiheit folgt \(k \cdot 1 = 0\) oder \(l \cdot 1\) = 0, und somit ein Widerspruch.

\subsubsection*{Definition Schiefk\"orper}
Ein K\"orper ohne kommutativit\"at in der Multiplikation nennen wir Schiefk\"orper. (Beispiel: Quaternionen. Siehe \"Ubungsblatt.)

\section{Vektorr\"aume}
Wir kennen bereits  \(\R^n = \R \times \R \times \dots \times \R\) mit Operationen \(a + b\) f\"ur \(a,b \in \R^n\) und \(\lb \cdot a\) f\"ur \(a \in \R^n, \lb \in \R\).

\subsection{Definitionen und elementare Eigenschaften}

\subsubsection{Definition 2.1 Vektorraum}
Es sei K ein K\"orper, \((V, +)\) eine abelsche Gruppe mit einer Abbildung \(K \times V \to V\), \((\lb, b) \mapsto \lb v\)\\
so dass gilt 
\begin{enumerate}
\item{\(\lb (x + y) = (\lb x) + (\lb y)\)}
\item{\((\lb + \mu) x = (\lb x) + (\mu x)\)}
\item{\(\lb (\mu x) = (\lb \mu) x\)}
\item{\(1 x = x\)}
\end{enumerate}
f\"ur alle \(x, y \in V\), \(\lb, \mu \in K\).\\
(Zu beachten ist hierbei, was Addition der Gruppe, was Multiplikation des K\"orpers und was die speziell definierte Abbildung ist. Dies ergibt sich jedoch eindeutig aus den Typen der Verkn\"upften Elemente.).\\
Wir nennen die Abbildung \((\lb, v) \mapsto \lb v\) skalare Multiplikation. Die Gruppe \((V, +)\) mit der skalaren Multiplikation  hei\ss{}t dann K-Vektorraum.\\
\\
\textbf{Bemerkung}
\begin{enumerate}
\item{Ist \((R, +, \cdot)\) ein Ring, \((V, +)\) eine abelsche Gruppe mit Abbildung \(R \times V \to V, (\lb, v) \mapsto \lb v\) welche die Bedingungen aus Def. 2,1 erf\"ullt. Dann ist V ein R-Modul (bzw. Links-R-Modul.).\\
Rechts-R-Moduln analog.}
\item{
\begin{itemize}
\item{Elemente in V hei\ss{}en Vektoren, Elemente in K hei\ss{}en Skalare.}
\item{Das Inverse zu \(a \in V\) hei\ss{}t -a (das Inverse f\"ur Gruppen mit Addition)}
\end{itemize}
}
\item{
Wir schreiben \((\lb x) + (\mu y) = \lb x + \mu y\) (d.h. ''Punkt vor Strich'' f\"ur skalare Multiplikation)
}
\item{
	\(K = \R\): reelle Vektorr\"aume.\\
	\(K = \C\) : komplexe Vektorr\"aume.
}
\end{enumerate}
\textbf{Beispiele}
\begin{enumerate}
\item{\mRn, siehe Kap 0.}
\item{\(\C^n, K = \C\) analog.}
\item{Sei K ein beliebiger K\"orper, dann ist \(K^n\) ein Vektorraum, der aus den n-Tupeln von K\"orperelementen besteht. Addition in \(K^n\) eintragweise, Multiplikation f\"ur \(\lb \in K\) ebenfalls eintragweise.\\
\(
\v{v_1\\ \vdots\\ v_n} + \v{w_1\\ \vdots\\ w_n} = \v{v_1 + w_1\\ \vdots \\ v_n + w_n}\\
\lb \v{w_1\\ \vdots\\ w_n} = \v{\lb w_1\\ \vdots \\ \lb w_n}
\)\\
\(K^0 := \{0\}\) ist der triviale Vektorraum.}
\item{Es sei K ein K\"orper, X eine Menge, \(V = Abb(X, K)\) mit\\
\((f + g)(x) = f(x) + g(x)\) f\"ur alle \(x \in X, f, g \in V\).\\
Damit wird V zu einer abelschen Gruppe, denn oben ist eine Addition \(+(f, g)\) definiert.\\
Wir definieren nun \((\lb f)(x) = \lb (f(x))\) f\"ur alle \(\lb \in K, f \in V, x \in X\) als Skalarmultiplikation.\\
Damit wird V zu einem Vektorraum.
}
\end{enumerate}

\subsubsection{Proposition 2.2 Eigenschaften von Vektorr\"aumen}
Es sei V eine K-Vektorraum. Dann gilt
\begin{enumerate}
\item{\(0 x = 0 \in V\) f\"ur alle \(x \in V\)}
\item{\(\lb 0 = 0\) f\"ur alle \(\lb \in K\)}
\item{Falls \(\lb \in K, x \in V, \lb x = 0 \in V\), dann gilt \(\lb = 0\) oder \(x = 0\).}
\item{\((-1) x = -x\) f\"ur alle \(x \in V\), mit \(-1 \)}
\end{enumerate}
\textbf{Beweis}
\begin{enumerate}
\item{\(0 x = (0 + 0)x = 0x + 0x \Rightarrow 0x = 0\).}
\item{\(\lb 0 = \lb(0 + 0) = \lb 0 + \lb 0 \Rightarrow \lb 0 = 0\).}
\item{Zu zeigen ist \(\lb \in K^*, x \in V, \lb x = 0\) dann folgt \(x = 0\).\\
Es gilt aber \(x = 1 x \overset{\lb \neq 0}{=} (\lb ^{-1} \lb) x = \lb^{-1}(\lb x) = \lb^{-1} 0 = 0\)}
\item{\(x + (-1) x = 1x + (-1) x = (1 - 1)  x = 0 x = 0\)}
\end{enumerate}
\textbf{Bemerkung}\\
Es sei \((G. +)\) eine Gruppe, \(y \in G\). Falls gilt \(y = y + y\), so folgt \(y = 0\), denn die K\"urzungsregel besagt \(a + \hat{x} = a+x \Rightarrow x = \hat{x}\). Mit \(x = 0, \hat{x} = y, a = y\). Also \(y +y = y + 0 = y \Rightarrow y = 0\).

\subsubsection{Definition 2.3 Untervektorr\"aume}
Es sei K ein K\"orper, V ein K-Vektorraum. Weiteres sei \(W \subset V\). Dann hei\ss{}t W Untervektorraum von V, falls gilt
\begin{enumerate}
\item{\(W \neq \emptyset\)}
\item{\(v, w \in W \Rightarrow v + w \in W\)}
\item{\(v \in W, \lb \in K \Rightarrow \lb v \in W\)}
\end{enumerate}
\textbf{Beispiel}\\
\(V = R^2, W = \{v = (v_1, v_2) \in V : v_1 = 0\}\)\\
\textbf{Gegenbeispiel}\\
\(V = R^2, W = \{v = (v_1, v_2) \in V : v_2 = 1\}\) ist kein Untervektorraum von V.

\subsubsection{Satz 2.4 Untervektorr\"aume sind Vektorr\"aume}
Ein Untervektorraum ist (mit der induzierten Addition und Skalarmultiplikation) ein Vektorraum.\\
\textbf{Beweis}\\
Sei V ein K-Vektorraum, W ein Untervektorraum von V.
\begin{enumerate}
\item{W ist eine Untergruppe von \((V, +)\), denn W ist nicht leer, abgeschlossen bez\"uglich der Addition. Das neutrale Element \(0 \in W\), denn f\"ur ein beliebiges \(w \in W\) folgt mit 3., dass \(0 = 0w \in W\). Zu \(v \in W\) gilt weiter \(-v = (-1)v \in W\) nach 3..}
\item{Kommutativit\"at und Assoziativit\"at der Untergruppe \((W, +)\) folgt sofort, Distributivgesetze ebenfalls.}
\end{enumerate}
Damit ist W ein Vektorraum.\\
\\
\textbf{Bemerkung} zur Notation:\\
Es sei I eine Menge und f\"ur jedes \(a \in I\) sei \(M_a\) wieder eine Menge. So ein I nennen wir Indexmenge. Nun verallgemeinern wir.Schnittmengen, etc.\\
\(\bigcap_{a\in I} M_a = \{x : x \in M_a \text{ f\"ur jedes\ } a \in I\}\)\\
\(\bigcup_{a\in I} M_a = \{x : x \in M_a \text{ f\"ur ein\ } a \in I\}\)

\subsubsection{Lemma 2.5}
Es sei V ein K-Vektorraum, I eine Indexmenge und f\"ur jedes \(a \in I\) sei \(W_a \subset V\) ein Untervektorraum. Dann gilt
\begin{enumerate}
\item{\(W = \bigcap_{a \in I} W_a\) ist ein Untervektorraum von V.}
\item{Seien \(a, b \in I\), dann folgt \(\hat{W} = W_a \ \cup W_b\) ist ein Untervektorraum von V genau dann, wenn \(W_a \subset W_b\) oder \(W_b \subset W_a\)}
\end{enumerate}
\textbf{Beispiele}\\
\(V = R^3, I = \{1, 2\},\\
W_1 = \{v = (v_1, v_2, v_3) \in V : v_1 = 0\},\\
W_2 = \{v = (v_1, v_2, v_3) \in V : v_2 = 0\}\\
W = w_1 \cap W_2 = \{v = (v_1, v_2, v_3) \in V : v_1 = v_2 = 0\}\) ist ein Untervektorraum.\\
\(W_1 \cup W_2 = \{v = (v_1, v_2, v_3) \in V : v_1 = 0 \lor v_2 = 0\}\) ist kein Untervektorraum von V, denn \(w_1 = (0, 1, 1) \in W_1, w_2 = (1, 0, 1) = W_2\), aber \(w_1 + w_2 = (1, 1, 2)\) ist nicht in \(W_1 \cup W_2\).\\
\\
\textbf{Beweis}
\begin{enumerate}
\item{Es gilt \(0 \in W_a\) f\"ur jedes \(a \in I\), also gilt \(0 \in W\)\\
Es seien \(x, y \in W\), also gilt \(x, y \in W_a\) f\"ur jedes \(a \in I\). Nachdem \(W_a\) (f\"ur jedes a) ein Untervektorraum von V ist, gilt \(x + y \in W_a\) f\"ur jedes \(a \in I\), also \(x + y \in W\). Ebenso folgt \(\lb x \in W\).}
\item{''\(\Leftarrow\)'' folgt sofort, denn wenn \(W_a \subset W_b\), so gilt \(W_a \cup W_b = W_b\) und somit \(W = W_b\) UVR (Untervektorraum).\\
''\(\Rightarrow\)'' Sei \(\hat{W} = W_a \cup W_b \subset V\) ein UVR und sei \(W_a \not\subset W_b\). Zu zeigen ist nun, \(W_b \subset W_a\). Es sei \(x \in W_b\), wir zeigen, dass folgt \(x \in W_a\). Sei \(y \in W_a \setminus W_b\) (so ein y existiert, nachdem \(W_a \not\subset W_b\)). Es folgt \(x + y \in \hat{W}\), also \(x + y \in W_a\) oder \(x + y \in W_b\). Es gilt aber, dass \(y = (x + y) - x\), und somit \(x + y\not\in W_b\). Somit gilt \(x + y \in W_a\), also \((x + y) - y = x \in W_a\).}
\end{enumerate}

\subsubsection{Definition 2.6 Linearkombination und Erzeugendensysteme}
Es sei V ein K-Vektorraum, \(E \subset V\) eine Menge.
\begin{enumerate}
\item{F\"ur jedes \(e \in E\) sei \(\lb_e \in K\), so dass nur endlich viele \(\lb_e \neq 0\) sind. Dann schreiben wir \(\sum_{e \in E}\lb_e \cdot e = \sum_{e\in E, \lb_e \neq 0} \lb_e \cdot e \in V\) und \(\sum_{e \in V} \lb_e \cdot e\) hei\ss{}t Linearkombination der \(e \in E\).}
\item{\(x \in V\) hei\ss{}t darstellbar als Linearkombination der \(e \in E\), falls \(\lb_e \in k\) existeiren, mit \(\lb_e \neq 0\) f\"ur endlich viele \(e \in E\) und es gilt \(x = \sum_{e \in E}\lb_e \cdot e\).}
\item{\(span(E) = \{x \in V : x \text{ als Linearkombination der } e \in E \text{ darstellbar}\}\)}
\item{Falls gilt \(W = span(E)\), so hei\ss{}t \(E \subset V\) Erzeugendensystem von W.}
\item{\(W \subset V\) hei\ss{}t endlich erzeugt, \"uber K, falls ein Erzeugendensystem f\"ur W mit nur endlich vielen Elementen existiert.}
\end{enumerate}
\textbf{Beispiel}\\
\(V = R^2, E = \{(1, 0), (0, 1), (1, 1)\} \subset V\). Dann gilt \(V = span(E)\), denn sei \(v = (v_1, v_2) \in R^2, v_1, v_2 \in \R\) und es gilt \(v = v_1 + \cdot (1, 0) + v_2 \cdot (0, 1)\). V ist also endlich erzeugt.

\subsubsection{Lemma 2.7}
Es sei V ein k-Vektorraum, \(E \subset V\). Dann gilt
\begin{enumerate}
\item{\(span(E)\) ist ein UVR von V.}
\item{Falls \(W \subset V\) ein UVR ist mit \(E \subset W\), so gilt \(span(E) \subset W\). (Es folgt ,dass \(span(E) \subset V\) der minimale Untervektorraum ist, der E enth\"alt.)}
\end{enumerate}
\textbf{Beweis}
\begin{enumerate}
\item{Folgt sofort aus der Definition, denn 
\begin{enumerate}
\item{\(span(E) \neq \emptyset\), denn \(0 \in span(E)\)}
\item{F\"ur \(v_1, v_2 \in span(E)\) gilt \(v_1 + v_2 \in span(E)\), denn wir k\"onnen die Koeffizienten \(\lb_e^{v1}\) und \(\lb_e^{v2}\) addieren. Ebenso f\"ur \(\mu v_1\).}
\end{enumerate}}
\item{Sei \(W \subset V\) ein Untervektorraum, \(E \subset W\). Es folgt sofort, dass (wegen Abgeschlossenheit von W bez\"uglich der Addition und Skalarmultiplikation), dass jede Linearkombination der \(e \in E\) wieder in W liegt.}
\end{enumerate}

\subsection{Basis und Dimension}
\textbf{Ziel}:\\
finde m\"oglichst kleine Erzeugendensysteme f\"ur Vektorr\"aume.\\
\textbf{Beispiel:}\\
\(R^2, e_1=(1, 0), e_2 = (0,1)\) dann gilt \(span(\{e_1, e_2\})\\
= \{x \in \R^2, x = (x_1, x_2), x_1 = \lb_1 e_1, x_2 = \lb_2 e_2, \lb_1, \lb_2 \in \R\} = \R^2\). Aber nur \(\{e_1\}\) oder \(\{e_2\}\) ist kein Erzeugendensystem f\"ur \(\R^2\).\\
Mit \(e_3 = \{1, 1\}\) ist \(\{e_1, e_2, e_3\}\) ein Erzeugendensystem f\"ur \(\R^2\), aber kein kleinstm\"ogliches im obigen Sinne.\\
\textit{Im folgenden sei stets} \textbf{K ein K\"orper} \textit{und} \textbf{V ein K-Vektorraum}

\subsubsection{Definition 2.8 Familien von Elementen} 
Seien X, I Mengen. F\"ur jedes \(j \in I\) sei \(e_j \in X\). Dann berechnen wir die Abbildung \(I \to X, j \mapsto e_j\) als ''durch I induzierte Familie von Elementen von X''. Wir schreiben \((e_j)_{j \in I} \in X^I = Abb(I, X)\).\\
\\
\textbf{Bemerkung}\\
Es sei \(I = \{1, 2, 3, \dots, n\}\). Dann gilt \(\R^I = \R^{\{1, 2, \dots, n\}} = \R^n\). Die Abbildung ist hier \(1 \mapsto x_1, 2 \mapsto x_2, \dots, n \mapsto x_n\).
\\
F\"ur \(I = \N\) is \(\R^\N\) die Menge der reellen Folgen.

\subsubsection{Definition 2.9 Minimale und linear unab\"angige Erzeugendensysteme}
Es sei I eine Menge, und \((v_i)_{i \in I}\) eine Familie von Vektoren \(v_i \in V\).
\begin{enumerate}
\item{\((v_i)_{i \in I}\) hei\ss{}t minimales Eerzeugendensystem von V, falls \(E = \{v_i, i \in I\}\) ein Erzeugendensystem von V ist und gilt \((J \subsetneq I) \Rightarrow span(v_j, j \in J) \neq V\)}
\item{\((v_i)_{i \in I}\) hei\ss{}t lineare unabh\"angig, falls gilt:\\
Es sei \((\lb_i)_{i \in I} \in K^I\) (eine durch I induzierte Menge von Skalaren) mit \(\lb_i \neq 0\) f\"ur endlich viele \(i \in I\) und \(\sum_{i \in I} \lb_i v_i = 0\), dann folgt \(\lb_i = 0\) f\"ur alle \(i \in I\). Nicht linear unabh\"angige Familien hei\ss{}en lineare abh\"angig.\\
\textbf{Bemerkung}: F\"ur \(I = \emptyset\) ist \((v_i)\_{i \in I}\) stets linear unabh\"angig.}
\end{enumerate}
\textbf{Beispiele}\\
Siehe Kapitel 0.

\subsubsection{Lemma 2.10}
Es sei \((v_i)_{i \in I} \in V^I\). Dann gilt:
\begin{enumerate}
\item{Falls \(v_j = 0\) f\"ur ein \(j \in I\), dann ist \((v_i)_{i \in I}\) linear abh\"angig.}
\item{Falls \(i, j \in I\) existieren, mit \(v_i = v_j\) dann ist \((v_i)_{i \in I}\) linear abh\"angig.}
\item{Falls \(I = \{i\}\) ist, dann ist \((v_i)_{i \in I}\) linear unabh\"angig genau dann, wenn \(v_i \neq 0\)}
\item{\((v_i)_{i \in I}\) linear unabh\"angig, dann gilt \((J \subset I) \Rightarrow (v_j)_{j \in J}\) ist ebenfalls linear unabh\"angig.}
\item{Sei \(I \neq \emptyset\), dann gilt \((v_i)_{i \in I}\) ist lineare abh\"angig genau dann, wenn \(j_0 \in I\) existiert so dass \(J \subset I \setminus \{j_0\}\), J ist eine endliche Mengen und \((\mu_j)_{j \in J} \in K^I\) existiert, so dass \(\mu_{j_0} = \sum_{j \in J} \mu_j v_j \).\\
(Ein Element l\"asst sich als Linearkombination der anderen schreiben).}
\end{enumerate}
\textbf{Beweis}
\begin{enumerate}
\item{\(1 v_j = 0\) und \(1 \neq 0\)}
\item{Klar aus i) und Definition}
\item{Folgt direkt aus der Definition.}
\item{''linear abh\"angig'' \(\Rightarrow\) es ex. ''linear Koeffizienten''.\\
Sei also \((v_i)_{I \in I}\) lin. abh\"angig \(\Rightarrow \exists (\lb_i)_{i\in I} \in K^I\) (nur endlich viele \(\neq 0\)) und ein \(\lb_{i_0} \neq 0\) mit \(i_0 \in I\) und\(\sum_{i \in I} \lb_i v_i = 0\). Damit gilt \(\lb_{i_0} v_{i_0} = - \sum_{i \in I}\lb_i v_i\\
\Rightarrow v_{i_0} = -\sum_{i \in I \setminus \{i_0\}} \frac{\lb_i}{\lb_{i_0}} v_i\)\\
''\(\Leftarrow\)''\\
Es sei \(v_{j_0} = \sum_{i \in I \setminus \{i_0\}} \mu_i v_i\) (wie in Vorraussetzung), dann setzen wir\\
\(\lb_i = \begin{cases}1, & \text{falls } i = j_0\\-\mu_i & \text{falls } i\in I \setminus j_0\end{cases}\)\\
und es gilt \(\sum_{i \in I} \lb_i v_i = 0\).}
\end{enumerate}

\subsubsection{Satz 2.11}
Es sei \((v_i)_{i\in I} \in V^I\). Dann sind \"aquivalent:
\begin{enumerate}
\item{\((v_i)_{i \in I}\) ist ein miniales Erzeugendensystem von V.}
\item{\((v_i)_{i \in I}\) ist ein linear unabh\"angiges Erzeugendensystem von V}
\item{jedes \(v \in V\) besitzt eine eindeutige Darstellung als Linearkombination der \(v_i\) mit \(i \in I\)}
\item{\((v_i)_{i \in I}\) ist eine maximale lienare unabh\"angige Familie, d.h. f\"ur jedes \(w \in V\) gilt \((w, (v_i)_{i \in I})\) ist linear abh\"angig.}
\end{enumerate}
\textbf{Beweis}\\
Zu zeigen \(1 \Rightarrow 2 \Rightarrow 3 \Rightarrow 4 \Rightarrow 1\) (Zirkelschluss).\\
\begin{itemize}
\item[\(1 \Rightarrow 2\)]{Beweis durch Widerspruch: \((\neg 2 \Rightarrow \neg 1)\)\\
Sei also \((v_i)_{i \in I}\) ein minimales Erzeugendensystem, aber linear abh\"angig. Es gilt aus Lemma 2.10(5), dass gilt: es ex. ein \(j_0 \in I\), mit \(v_{j_0} = \sum_{i \in I \setminus \{j_0\}} \lb_i v_i\) (\(\lb_i \in K \) geeignet). Also gilt \(v_{j_0} \in span(\{v_i : i \in I \setminus \{j_0\}\})\). Mit Lemma 2.7 folgt aber, wenn \(W, \hat{W}\) UVR on V, \(W \cup \hat{W}\) UVR von V.\\
\(\Rightarrow W \subset \hat{W}\) oder \(\hat{W} \subset W\)\\
\(\Rightarrow span(\{v_i, i \in I\}) = span(\{v_i, i \in I \setminus \{j_0\}\} = V\). Damit ist \(\{v_i, i \in I\}\) kein minimales Erzeugendensystem und wir erhalten einen Widerspruch.
}
\item[\(2 \Rightarrow 3\)]{ \(v \in V \Rightarrow v = \sum_{i \in I}\lb_i v_i\) (\(\lb_i\) geeignet). Angenommen die Darstellung sei nicht eindeutig, d.h. \(v = \sum_{i \in I}\bar{\lb_i} v_i\), dann gilt \(v - v = 0 = \sum_{i \in I} (\lb_i - \bar{\lb_i}) v_i\), aus 2 folgt \(\lb_i - \bar{\lb_i} = 0 \Rightarrow \lb_i = \bar{\lb_i}\), somit ist die Darstellung eindeutig.
}
\item[\(3 \Rightarrow 4\)]{Es sei eine eind. Darstellung f\"ur jedes \(v \in V\). Zu zeigen ist:
\begin{enumerate}
\item[a.)]{\((v_i)_{i \in I}\) ist linear unabh\"angig}
\item[b.)]{F\"ur jedes \(w \in V\) ist \((w, (v_i)_{i \in I})\) linear abh\"angig}
\end{enumerate}
Zu a.)\\
Es sei \(\sum_{i \in I}\lb_i v_i = 0\) (\(\lb_i\) geeignet). Es gilt \(0 \in V\), die Darstellung \(0 = \sum_{i \in I}\mu_i v_i\) mit \(\mu_i = 0\) f\"ur jedes \(i \in I\) ist nach 3 eindeutig, also folgt \(\lb_i = \mu_i = 0 \forall i \in I\).\\
Zu b.)\\
Sei \(w \in V, w = \sum_{i \in I} \lb_i v_i\) (\(\lb_i\) geeignet). Dann gilt aber mit Lemma 2.10, dass \((w, (v_i)_{i \in I})\) linear abh\"angig ist.
}
\item[\(4 \Rightarrow 1\)]{
Sei \((v_i)_{i \in I}\) eine maximale lineare unabh\"angige Familie. Zu zeigen ist
\begin{itemize}
\item[a.)]{\(span(\{v_i, i \in I\}) = V\)}
\item[b.)]{\(span(\{v_j, j \in J\}) \neq V\) f\"ur \(J \subsetneq I\).}
\end{itemize}
zu a.)\\
Falls w  \(\in V \setminus span(\{v_i, i \in I\})\) existiert ist \((w, (v_i)_{i \in I})\) linear unabh\"angig, denn \(\mu_w + \sum_{i \in I} \lb_i v_i = 0\) (mit \(\lb_i, \mu \in K\) geeignet) impliziert \(\mu = 0\) und damit \(\lb_i = 0\) f\"ur alle \(i \in I\). Das steht im Widerspruch zu der maximalen Unabh\"anigkeit der Familie.\\
zu b.)\\
Es sei \(V = span(\{v_j : j \in J\}), J \subsetneq I\). Dann gilt \(v_{j_0}\) mit \(j_0 \in I \setminus J\) l\"asst sich als Linearkombination \(v_{j_0} = \sum_{j \in J} \lb_j v_j\) schreiben, ein Widerspruch zur linearen Unabh\"angigkeit der \(v_j\).
}
\end{itemize}
\end{document}
